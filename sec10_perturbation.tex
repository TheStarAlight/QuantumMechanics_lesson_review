\section{微扰论}
\label{sec:perturbation}

在任何学科中,能够精确、完全解析处理的问题少之又少,在量子力学中也是如此。
微扰作为一种重要的近似方法,被广泛应用在量子力学各类问题的研究中,这类问题都有一个特点:由主体部分和``微扰''(perturbation)部分组成,其主体部分拥有精确解,而微扰部分相比主体部分微弱得多。
微扰方法的关键思想就是:将微扰部分对系统带来的影响用``修正''(correction)表示,用一个渐近级数来逼近真实的解。

% =================================================
\subsection{非简并定态微扰}
\label{subsec:non_deg_stationary_perturbation}

我们假定系统的Hamiltonian为
\begin{equation}
    H = H_0 + λH',
\end{equation}
其中$H_0$为未扰Hamiltonian,$λH'$为微扰Hamiltonian,$λ$作为一个小的参数。
未扰体系有精确解$\ket{m^{(0)}}$,满足
\begin{equation}
    H_0\ket{m^{(0)}} = E^{(0)}_m\ket{m^{(0)}}.
\end{equation}

在微扰论中,我们将对态$\ket{m}$和能量$E_m$的修正表为关于$λ$的级数:
\begin{equation}
\begin{aligned}
    \ket{m} &= \ket{m^{(0)}} + \ket{m^{(1)}}λ + \ket{m^{(2)}}λ^2 + ⋯,\\
    E_m &= E_m^{(0)} + E_m^{(1)}λ + E_m^{(2)}λ^2 + ⋯.
\end{aligned}
\end{equation}
将这一级数代入Schrödinger方程,可得满足的关系:
\begin{equation}
    (H_0+λH')\left(\ket{m^{(0)}} + \ket{m^{(1)}}λ + ⋯\right) = \left(E_m^{(0)} + E_m^{(1)}λ + ⋯\right) \left(\ket{m^{(0)}} + \ket{m^{(1)}}λ + ⋯\right).
\end{equation}
这一式等号两侧可以分别整理为$λ$的级数,每一阶的系数都相同,就分别得到一个方程。

\noindent
\fbox{\textbf{ 零阶 }}
$λ$的零阶方程就对应未扰系统的Schrödinger方程
\begin{equation}
    H_0\ket{m^{(0)}} = E^{(0)}_m\ket{m^{(0)}}.
\end{equation}

\noindent
\fbox{\textbf{ 一阶 }}
$λ$的一阶方程为
\begin{equation}
    \label{eq:perturb_lambda_first_order}
    H'\ket{m^{(0)}}+H_0\ket{m^{(1)}} = E^{(1)}_m\ket{m^{(0)}}+E^{(0)}_m\ket{m^{(1)}}.
\end{equation}
移项后,可得
\begin{equation}
    (H'-E_m^{(1)})\ket{m^{(0)}} + (H_0-E_m^{(0)})\ket{m^{(1)}} = 0.
\end{equation}
如果我们插入一组能量本征态$\ket{n^{(0)}}$(具有本征值$E_n^{(0)}$)的完备性关系
\begin{equation}
    \mathbb{1} = ∑_n \ketbra{n^{(0)}},
\end{equation}
依$n=m$和$n≠m$分开,并利用$\ket{n^{(0)}}$是正交归一的能量本征态,上式就成为
\begin{equation}
\begin{aligned}
    & \left[\mel{m^{(0)}}{H'}{m^{(0)}} + (E_m^{(0)}-E_m^{(0)})\braket{m^{(0)}}{m^{(1)}} - E_m^{(1)}\right]\ket{m^{(0)}} \\
    +& ∑_{n≠m} \left[\mel{n^{(0)}}{H'}{m^{(0)}} + (E_n^{(0)}-E_m^{(0)})\braket{n^{(0)}}{m^{(1)}}\right]\ket{n^{(0)}} = 0.
\end{aligned}
\end{equation}

我们先看$n=m$的情形,给出
\begin{equation}
    \underbrace{\mel{m^{(0)}}{H'}{m^{(0)}}}_{H'_{mm}} + (E_m^{(0)}-E_m^{(0)})\braket{m^{(0)}}{m^{(1)}} - E_m^{(1)} = 0.
\end{equation}
这里需要分清哪些是待定量,哪些是已知量:$H'_{mm}$和$E_m^{(0)}$是独立的已知量,而$E_m^{(1)}$和$\braket{m^{(0)}}{m^{(1)}}$(即$\ket{m^{(1)}}$投影在$\ket{m^{(0)}}$的分量)是待定量。
这个式子保留了$(E_m^{(0)}-E_m^{(0)})=0$,就是因为$\braket{m^{(0)}}{m^{(1)}}$也是待定量,只是由于其系数为0,无法从本式中确定其取值。
\emph{一阶能量修正}$E_m^{(1)}$为
\begin{equation}
    \label{eq:perturb_energy_corr_1}
    E_m^{(1)} = H'_{mm}.
\end{equation}

再看$n≠m$的情形,给出
\begin{equation}
    \underbrace{\mel{n^{(0)}}{H'}{m^{(0)}}}_{H'_{nm}} + (E_n^{(0)}-E_m^{(0)})\braket{n^{(0)}}{m^{(1)}} = 0.
\end{equation}
通过该式可以求出\emph{态的一阶修正}
\begin{equation}
    \braket{n^{(0)}}{m^{(1)}} = \frac{H'_{nm}}{E_m^{(0)}-E_n^{(0)}}\quad (n≠m).
\end{equation}
可见,为使这里的一阶修正不发散,须假定体系没有简并能级。
我们可以这样理解:
加入了微扰项$λH'$后,$\ket{m^{(0)}}$不再是新Hamiltonian $H_0+λH'$的本征态,而是混入了其他未扰能级$\ket{n^{(0)}}$的成分,这些成分将使能级发生偏移。

最后我们还需确定$\braket{m^{(0)}}{m^{(1)}}$,可以借助一阶近似下的态$\ket{m^{(0)}}+λ\ket{m^{(1)}}$的归一化条件
\begin{equation}
    \left(\bra{m^{(0)}}+λ\bra{m^{(1)}}\right)\left(\ket{m^{(0)}}+λ\ket{m^{(1)}}\right) = 1 + o(λ).
\end{equation}
显然,这要求
\begin{equation}
    \braket{m^{(0)}}{m^{(1)}} = 0.
\end{equation}

现在我们就得到\emph{态的一阶修正}:
\begin{equation}
    \label{eq:perturb_state_corr_1}
    \braket{n^{(0)}}{m^{(1)}} = \begin{cases}
        \frac{H'_{nm}}{E_m^{(0)}-E_n^{(0)}}    & n≠m,\\
        0,  & n=m.
    \end{cases}
\end{equation}

\noindent
\fbox{\textbf{ 二阶 }}
$λ$的二阶方程为
\begin{equation}
    H_0\ket{m^{(2)}} + H'\ket{m^{(1)}} = E_m^{(0)}\ket{m^{(2)}} + E_m^{(1)}\ket{m^{(1)}} + E_m^{(2)}\ket{m^{(0)}},
\end{equation}
移项后变为
\begin{equation}
    (H_0-E_m^{(0)})\ket{m^{(2)}} + (H'-E_m^{(1)})\ket{m^{(1)}} - E_m^{(2)}\ket{m^{(0)}} = 0.
\end{equation}

我们依然采用一阶的做法,上式左乘$\bra{m^{(0)}}$即得
\begin{equation}
    \mel{m^{(0)}}{H'}{m^{(1)}} - E_m^{(1)}\underbrace{\braket{m^{(0)}}{m^{(1)}}}_{=0} - E_m^{(2)} = 0,
\end{equation}
再借助态的一阶修正「式\eqref{eq:perturb_state_corr_1}」,可得\emph{能量二阶修正}:
\begin{equation}
    E_m^{(2)} = \mel{m^{(0)}}{H'}{m^{(1)}} = ∑_{n≠m}\frac{H'_{mn}H'_{nm}}{E_m^{(0)}-E_n^{(0)}} = ∑_{n≠m}\frac{\abs{H'_{mn}}^2}{E_m^{(0)}-E_n^{(0)}}.
\end{equation}
这里利用了$H'$是Hermite算符,满足$H'_{mn}=H'^*_{nm}$。

\emph{态的二阶修正}通过左乘$\bra{n^{(0)}}$($n≠m$)得到,可得其表式为
\begin{equation}
    \braket{n^{(0)}}{m^{(2)}} = \frac{\mel{n^{(0)}}{H'-E_m^{(1)}}{m^{(1)}}}{E_m^{(0)}-E_n^{(0)}}.
\end{equation}
通过展开$H'$与$\ket{m^{(1)}}$,可以得到更具体的表达式,此处不再赘述。


\begin{tcolorbox}[breakable, colframe=purple, colback=red!10, title={\textbf{高阶微扰的一般表式}}]
\it\small
从一阶和二阶修正的表达式中,我们已经可以看出一些规律。
现在我们来推导一下高阶微扰修正所应满足的方程。

我们写出$λ$的$k$阶方程
\begin{equation}
    H_0\ket{m^{(k)}} + H'\ket{m^{(k-1)}} = ∑_{i=0}^{k} E_m^{(i)}\ket{m^{(k-i)}}.
\end{equation}
左乘$\bra{m^{(0)}}$,利用高阶态修正均不含$\ket{m^{(0)}}$,即得能量修正
\begin{equation}
    E_m^{(k)} = \mel{m^{(0)}}{H'}{m^{(k-1)}}.
\end{equation}
左乘$\bra{n^{(0)}}$($n≠m$),可得态修正
\begin{equation}
    \braket{n^{(0)}}{m^{(k)}} = \frac{\mel{n^{(0)}}{H'}{m^{(k-1)}}-∑_{i=1}^{k}E_m^{(i)}\braket{n^{(0)}}{m^{(k-i)}}}{E_m^{(0)}-E_n^{(0)}}.
\end{equation}

\end{tcolorbox}

% =================================================
\subsection{简并情形定态微扰}
\label{subsec:deg_stationary_perturbation}

我们在上一节所获得的微扰公式,其仅在无简并情形下适用。
对于存在简并态的情形,微扰修正公式中的分母$E_n^{(0)}-E_m^{(0)}$将出现等于零的情况,这就是说,原有基组的微扰结果发散。
对于这种情况,需另行选择另一组合适的基组。

我们假定一组能级简并态基组为$\{\ket{n^{(0)}}\}$($n=1,2,⋯,s_0$,其中$s_0$为简并度),其拥有相同的能级$E^{(0)}$。
这些简并态可以线性组合为新的基组,系数为$c^{(0)}_{mn}$:
\begin{equation}
    \ket{m^{(0)}} = ∑_{n} c^{(0)}_{mn}\ket{n^{(0)}}.
\end{equation}
在一阶微扰下,$\ket{m^{(0)}}$须满足$λ$的一阶方程「式\eqref{eq:perturb_lambda_first_order}」。
我们左乘原基组的态$\bra{n^{(0)}}$,因仅计及一阶能级修正,我们略去$\ket{m^{(1)}}$,得到:
\begin{equation}
    \mel{n^{(0)}}{H'-E_m^{(1)}}{m^{(0)}} = 0,
\end{equation}
这等价于
\begin{equation}
    \label{eq:perturb_pertHamil_eigen_eq}
    H'\ket{m^{(0)}} = E_m^{(1)}\ket{m^{(0)}}.
\end{equation}
该式实际上是一个关于$H'$的本征方程,其中$E_m^{(1)}$作为本征值,本征矢则是$\ket{m^{(0)}}$。
因此,$\ket{m^{0}}$应当选取为$H'$的各本征态(即选取$H'$为对角的表象),对应的本征值即为一阶能级修正$E_m^{(1)}$。

倘若无法直接看出$H'$的本征态,那就要通过具体计算来求解。
方程\eqref{eq:perturb_pertHamil_eigen_eq}有非平凡解$\ket{m^{(0)}}$的要求是
\begin{equation}
    \label{eq:perturb_secular_eq}
    \det(H'-E_m^{(1)}) = 0,
\end{equation}
这个方程称为\emph{久期方程(secular equation)}
\footnote{这个名字来源于天体物理。在使用经典力学微扰求解行星运动受周期性扰动从而产生一个``长期''的偏离效应时,也出现了这个方程。}。
在$\ket{n^{(0)}}$表象下求解久期方程,即得叠加系数$c_{mn}^{(0)}$,线性组合得到的$\ket{m^{(0)}}$就是一阶能级修正为$E_m^{(0)}$的未扰态。

一般来说,$H'$的引入将导致不同的一阶能级修正$E_m^{(1)}$(即久期方程\eqref{eq:perturb_secular_eq}没有重根),从而解除各态的简并。
若一阶能级修正仍不足以解除简并,则需在式\eqref{eq:perturb_lambda_first_order}中计及一阶态修正$\ket{m^{(1)}}$。


% =================================================
% 习题
\begin{tcolorbox}[breakable, title={\textbf{例题1}}]
    \it\small
    \fbox{ \textbf{题目} }
    (2020年考研 · 七)
    一个质量为$m$的粒子被限制在半径为$a$的圆周上。求:\\
    (1) 体系的能量本征态波函数;\\
    (2) 设$θ$为圆周上的角度,对于微扰$H'=λ\sin{2θ}$,求微扰对基态波函数的一级修正、对基态能级的二级修正;\\
    (3) 在(2)的情形下,求微扰对第一激发态能级的一级修正与相应的零级波函数。\\

    \fbox{ \textbf{解答} }

    (1)
    首先,我们写出这个体系的Hamiltonian,假定圆环的轴线与$z$轴重合,有两种写法:
    \begin{equation}
        H = \frac12 Jω^2 \quad\rm{or}\quad H = \frac{L_z^2}{2J},
    \end{equation}
    这里$J=ma^2$是转动惯量,$L_z$和$ω$分别是角动量和角速度。
    选取角动量表示Hamiltonian的好处在于其表式是我们熟知的$L_z=\ii\hbar\pd_θ$。
    如果是用角速度表示,那就需要通过与角动量的关系进行转换,或者通过量纲分析配凑。

    未扰体系的定态Schrödinger方程,在$θ$表象下为
    \begin{equation}
        \frac{1}{2J}L_z^2 ψ(θ) = -\frac{\hbar^2}{2J}\pd^2_{θθ} ψ(θ) = Eψ(θ).
    \end{equation}
    这个方程类似于圆环上的弦振动方程,有周期性边界条件,通解是
    \begin{equation}
        ψ(θ) = A\cos{nθ} + B\sin{nθ}.
    \end{equation}
    需要注意的是,$n=0$的情况需要特别讨论。
    我们写出归一化解为
    \begin{equation}
    \begin{aligned}
        ψ_0(θ) &= \frac{1}{\sqrt{2π}};\\
        ψ_{n\rm{c}}(θ) &= \frac{1}{\sqrtπ}\cos{nθ}, n=1,2,⋯;\\
        ψ_{n\rm{s}}(θ) &= \frac{1}{\sqrtπ}\sin{nθ}, n=1,2,⋯.\\
    \end{aligned}
    \end{equation}
    根据解的性质很容易得到对应的能量
    \begin{equation}
        E_n = \frac{n^2\hbar^2}{2J} = \frac{n^2\hbar^2}{2ma^2}.
    \end{equation}

    (2)
    基态是非简并的,适用非简并微扰论。

    在求解中,需要用到矩阵元$\mel{k}{H'}{0}$,其中$k=0,1\rm{c},1\rm{s},2\rm{c},2\rm{s},⋯$为态的指标。
    显然,根据三角函数的正交性,$\mel{k}{H'}{0}$仅有$k=2\rm{s}$这一项不为零:
    \begin{equation}
        \mel{0}{H'}{0} = 0, \quad
        \mel{k≠0}{H'}{0} = \frac{λ}{\sqrt{2}} δ_{k,2\rm{s}}.
    \end{equation}

    零级能量为$E_0^{(0)}=0$。
    一级能量修正为
    \begin{equation}
        E_0^{(1)} = \mel{0}{H'}{0} = 0.
    \end{equation}
    二级能量修正为
    \begin{equation}
        E_0^{(2)} = ∑_{k≠0}\frac{\abs{\mel{k}{H'}{0}}^2}{E_0^{(0)}-E_k^{(0)}} = \frac{\abs{\mel{2\rm{s}}{H'}{0}}^2}{E_0^{(0)}-E_2^{(0)}} = -\frac{Jλ^2}{4\hbar^2}.
    \end{equation}
    一级波函数修正为
    \begin{equation}
        \braket{k^{(0)}}{0^{(1)}} = \frac{\mel{k}{H'}{0}}{E_0^{(0)}-E_k^{(0)}} = -\frac{Jλ}{2\sqrt2\hbar^2}.
    \end{equation}

    (3)
    第一激发态是$k=1\rm{c},1\rm{s}$,是一对简并态,因此需要重新组合两个态作为零级波函数。
    久期方程表为
    \begin{equation}
        \begin{vmatrix}
            H'_{1\rm{c}1\rm{c}}-E^{(1)} & H'_{1\rm{c}1\rm{s}} \\
            H'_{1\rm{s}1\rm{c}} & H'_{1\rm{s}1\rm{s}}-E^{(1)}
        \end{vmatrix} = 0.
    \end{equation}
    求解$H'$的各矩阵元时,依然可以利用三角函数的正交性:
    在$H'_{1\rm{c}1\rm{c}}$和$H'_{1\rm{s}1\rm{s}}$中,波函数内积$\cos^2{θ}$、$\sin^2{θ}$均与$H'$对应的$\sin{2θ}$正交,因此两个矩阵元均为零。
    而
    \begin{equation}
        H'_{1\rm{c}1\rm{s}} = H'^*_{1\rm{s}1\rm{c}} = \mel{1\rm{c}}{H'}{1\rm{s}} = \fracλ2.
    \end{equation}
    因此久期方程为
    \begin{equation}
        \begin{vmatrix}
            -E^{(1)} & λ/2\\ λ/2 & -E^{(1)}
        \end{vmatrix} = 0,
    \end{equation}
    解为$E_a^{(1)}=λ/2, E_b^{(1)}=-λ/2$,对应的零级态分别为
    \begin{equation}
        ψ_a = \frac{1}{\sqrt2}\ket{1\rm{c}} + \frac{1}{\sqrt2}\ket{1\rm{s}},\quad
        ψ_b = \frac{1}{\sqrt2}\ket{1\rm{c}} - \frac{1}{\sqrt2}\ket{1\rm{s}}.
    \end{equation}
    \hfill $\square$
\end{tcolorbox}

\begin{tcolorbox}[breakable, title={\textbf{例题2}}]
    \it\small
    \fbox{ \textbf{题目} }
    (2021年考研 · 七)
    一个质量为$m$的粒子处于二维势场:
    \begin{equation}
        V(x,y) = \frac12 mω^2 (x^2+y^2) + λxy,
    \end{equation}
    其中$λ$是小的实参量。将$H'=λxy$视为微扰,求:\\
    (1) 基态、第一激发态、第二激发态的零级近似能量;\\
    (2) 基态、第一激发态、第二激发态的一级近似能量。\\
    提示:一维谐振子Fock表象的升降算符与坐标、动量算符间的变换关系:
    \begin{equation}
        x=\sqrt{\frac{\hbar}{2mω}}(a+\adag),\quad
        p=-\ii\sqrt{\frac{m\hbar ω}{2}}(a-\adag).
    \end{equation}

    \fbox{ \textbf{解答} }
    我们在Fock表象下求解这个问题。
    设$x$方向的玻色升降算符为$a,\adag$,$y$方向的玻色升降算符为$b,\bdag$。体系的未扰Hamiltonian表为
    \begin{equation}
        H_0 = \hbar ω (\adag a+\bdag b +1),
    \end{equation}
    微扰Hamiltonian为
    \begin{equation}
        H' = \frac{\hbar λ}{2mω} (a+\adag)(b+\bdag) = λ' (a+\adag)(b+\bdag).
    \end{equation}

    (1)
    未扰体系的能量本征态可以表为
    \begin{equation}
        \ket{m_x n_y} = \ket{m}_x ⊗ \ket{n}_y,\quad
        m,n = 0,1,2,⋯
    \end{equation}
    对应的能量是
    \begin{equation}
        E^{(0)}_{mn} = \hbar ω (m+n+1).
    \end{equation}
    于是很容易就得到:
    \begin{itemize}
        \item 基态:$\ket{00}$,对应能量$E^{(0)}=\hbar ω$;
        \item 第一激发态:$\ket{10},\ket{01}$,对应能量$E^{(0)}=2\hbar ω$;
        \item 第二激发态:$\ket{20},\ket{11},\ket{02}$,对应能量$E^{(0)}=3\hbar ω$。
    \end{itemize}

    (2)
    体系的基态不简并,因此一阶能量修正为
    \begin{equation}
        E^{(1)}_{00} = \mel{00}{H'}{00} = λ' \mel{0}{a+\adag}{0}\mel{0}{b+\bdag}{0} = 0.
    \end{equation}
    这里需要用到升降算符对Fock态的作用:
    \begin{equation}
        a\ket{n} = \sqrt{n}\ket{n-1},\quad
        \adag\ket{n} = \sqrt{n+1}\ket{n+1}.
    \end{equation}
    在求解$H'$的矩阵元时,注意我们选择的基底对于$H'$是可分离变量的,将$x$与$y$方向的算符分开计算可以大大简化繁琐的运算。

    第一激发态与第二激发态均简并,因此须求解久期方程。
    对于第一激发态,选取基底$\ket{10},\ket{01}$,久期方程表为
    \begin{equation}
        \begin{vmatrix}
            H'_{10,10}-E^{(1)} & H'_{10,01}\\
            H'_{01,10} & H'_{01,01}-E^{(1)}
        \end{vmatrix} = 0.
    \end{equation}
    算符$a+\adag$的特点是将$\ket{n}$变为$\ket{n-1}$与$\ket{n+1}$的线性组合,因此$H'$的对角元必定为零。
    两个非对角元是
    \begin{equation}
        H'_{01,10} = H'^*_{10,01} = λ'\mel{0}{a+\adag}{1}\mel{1}{b+\bdag}{0} = λ'\mel{0}{a}{1}\mel{1}{\bdag}{0} = λ'.
    \end{equation}
    于是得久期方程的解为
    \begin{equation}
        E_a^{(1)} = λ',\quad
        E_b^{(1)} = -λ',
    \end{equation}
    分别对应
    \begin{equation}
        \ket{ψ_a^{(0)}} = \frac{1}{\sqrt2}\ket{10}+\frac{1}{\sqrt2}\ket{01},\quad
        \ket{ψ_b^{(0)}} = \frac{1}{\sqrt2}\ket{10}-\frac{1}{\sqrt2}\ket{01}.
    \end{equation}

    对于第二激发态,选取基底$\ket{20},\ket{11},\ket{02}$。
    我们先来研究$H'$的矩阵元,考虑到$(a+\adag)$算符的特性,应当只有$(20,11)$和$(11,02)$之间$H'$的矩阵元不为零,我们求取
    \begin{equation}
    \begin{aligned}
        H'_{20,11} &= H'^*_{11,20} = λ'\mel{2}{\adag}{1}\mel{0}{b}{1} = \sqrt2 λ';\\
        H'_{11,02} &= H'^*_{02,11} = λ'\mel{1}{\adag}{0}\mel{1}{b}{2} = \sqrt2 λ'.
    \end{aligned}
    \end{equation}
    于是久期方程成为
    \begin{equation}
        \begin{vmatrix}
            -E^{(1)} & \sqrt2 λ'\\
            \sqrt2 λ'& -E^{(1)} & \sqrt2 λ'\\
                     & \sqrt2 λ'& -E^{(1)}
        \end{vmatrix} = 0,
    \end{equation}
    根据三阶行列式的求和规则(对角线法),久期方程等价于
    \begin{equation}
        -{E^{(1)}}^3 + 4E^{(1)}λ'^2 = 0,
    \end{equation}
    方程的解为
    \begin{equation}
        E_c^{(1)} = 0, E_d^{(1)} = 2λ', E_e^{(1)} = -2λ'.
    \end{equation}
    对应的本征矢为
    \begin{equation}
        c→\begin{bmatrix}1/\sqrt2\\0\\-1/\sqrt2\end{bmatrix},
        d→\begin{bmatrix}1/2\\1/\sqrt2\\1/2\end{bmatrix},
        e→\begin{bmatrix}1/2\\-1/\sqrt2\\1/2\end{bmatrix}.
    \end{equation}
    对应零阶态此处从略不表。
    \hfill $\square$

\end{tcolorbox}

% =================================================
\subsection{含时微扰}
\label{subsec:time_dependent_perturbation}

现在我们来研究微扰项含时的情形。
在Hamiltonian含时的情况下,能量并不守恒,因此也没有能级修正一说,我们只是对量子态作近似计算。

我们采用与\ref{subsec:non_deg_stationary_perturbation}节类似的做法。
我们假定微扰Hamiltonian为$H_0+λH'(t)$,原Hamiltonian $H_0$的本征态为$\ket{ψ^{(0)}}$。
微扰的解$\ket{ψ(t)}$满足的Schrödinger方程为
\begin{equation}
    \label{eq:perturb_timedep_schrodinger_mt}
    \ii\hbar\pd_t \ket{ψ(t)} = (H_0+λH'(t)) \ket{ψ(t)}.
\end{equation}
在含时微扰中,一般选择未扰情况下自由演化的态
\begin{equation}
    \ket{m^{(0)}(t)} = \ket{m^{(0)}}\ee^{-\ii ω_m t}
\end{equation}
作为基底(为方便起见我们设$ω_m:=E_m/\hbar$),考虑投影系数
\begin{equation}
    a_m(t) := \braket{m^{(0)}(t)}{ψ(t)}
\end{equation}
随时间的变化。
这样,我们可以将式\eqref{eq:perturb_timedep_schrodinger_mt}写成关于$a_m(t)$的方程:
\begin{equation}
    \ii\hbar\pd_t ∑_m \left(a_m(t)\ket{m^{(0)}(t)}\right) = ∑_m a_m(t)(H_0+λH'(t))\ket{m^{(0)}(t)}.
\end{equation}
利用$\ii\hbar\pd_t\ket{m^{(0)}(t)}=H_0\ket{m^{(0)}(t)}$,化简为
\begin{equation}
    \ii\hbar ∑_m \dot{a}_m(t)\ket{m^{(0)}(t)} = λ ∑_m a_m(t) H'(t)\ket{m^{(0)}(t)}.
\end{equation}
再左乘$\bra{n^{(0)}(t)}$,得到
\begin{equation}
    \ii\hbar\dot{a}_n(t) = λ ∑_m a_m(t) \mel{n^{(0)}(t)}{H'(t)}{m^{(0)}(t)}.
\end{equation}
我们置$ω_{nm}:=ω_n-ω_m$,以及
\begin{equation}
    \mel{n^{(0)}(t)}{H'(t)}{m^{(0)}(t)}
    = \mel{n^{(0)}}{H'(t)}{m^{(0)}}\ee^{\ii ω_{nm}t}
    = H'_{nm}(t)\ee^{\ii ω_{nm}t}
    = \tilde{H}'_{nm}(t),
\end{equation}
于是方程化为
\begin{equation}
    \ii\hbar\dot{a}_n(t) = λ ∑_m \tilde{H}'_{nm}(t) a_m(t),
\end{equation}
写成矩阵形式则为:
\begin{equation}
    \ii\hbar\dot{\bm{a}}(t) = λ \tilde{\bm{H}}'(t) \bm{a}(t).
\end{equation}
我们将$\bm{a}(t)$表为$λ$的级数:
\begin{equation}
    \bm{a}(t) = \bm{a}^{(0)}(t) + \bm{a}^{(1)}(t)λ + \bm{a}^{(2)}(t)λ^2 + ⋯
\end{equation}
代入上式,即得关于$λ$的各阶方程。

$λ$的零阶方程为
\begin{equation}
    \ii\hbar\dot{\bm{a}}^{(0)}(t) = \bm{0}.
\end{equation}
这说明零阶近似下$\bm{a}^{(0)}(t)$为常数。

$λ$的一阶方程为
\begin{equation}
    \ii\hbar\dot{\bm{a}}^{(1)}(t) = \tilde{\bm{H}}'(t) \bm{a}^{(0)}(t).
\end{equation}
在已知初条件$\bm{a}^{(1)}(t=-∞)=0$的情况下,这一方程的解为
\begin{equation}
    \label{eq:perturb_timedep_first_order_mat}
    \bm{a}^{(1)}(t) = -\frac{\ii}{\hbar}∫_{-∞}^t \tilde{\bm{H}}'(τ) \bm{a}^{(0)}(τ) \dd{τ}.
\end{equation}
我们可以把这一表达式写成分量形式:
\begin{equation}
    \label{eq:perturb_timedep_first_order}
    a_n^{(1)}(t) = -\frac{\ii}{\hbar}∑_m ∫_{-∞}^t \tilde{H}'_{nm}(τ) a_m^{(0)}(τ) \dd{τ} = -\frac{\ii}{\hbar}∑_m ∫_{-∞}^t H'_{nm}(τ) a_m^{(0)}(τ) \ee^{\iiω_{nm}τ} \dd{τ},
\end{equation}
上式中,左边是Heisenberg绘景的惯常写法(将时间演化的相位吸收进Hamiltonian算符中),右边则是Schrödinger绘景的。
可见一阶微扰「式\eqref{eq:perturb_timedep_first_order}」描述了一个$τ$时刻因$H'$的存在从$m$态到$n$态的\emph{``跃迁''(transition)},其贡献由不同$τ$时刻和不同的初态$m$累加而来。
跃迁的概率与$H'$的强度直接相关,假如$H'(t)$变为两倍,即$2H'(t)$,那么跃迁概率幅$a^{(1)}(t)$将变为原来的两倍,表征跃迁概率的$\abs{a^{(1)}(t)}$将变为原来的四倍。

$λ$的二阶方程为
\begin{equation}
    \ii\hbar\dot{\bm{a}}^{(2)}(t) = \tilde{\bm{H}}'(t) \bm{a}^{(1)}(t).
\end{equation}
显然,方程的解为
\begin{equation}
    \bm{a}^{(2)}(t) = -\frac{\ii}{\hbar}∫_{-∞}^t \tilde{\bm{H}}'(τ) \bm{a}^{(1)}(τ) \dd{τ}.
\end{equation}
将一阶微扰的结果「式\eqref{eq:perturb_timedep_first_order_mat}」代入上式,有
\begin{equation}
    \label{eq:perturb_timedep_second_order_mat}
    \bm{a}^{(2)}(t) = \left(-\frac{\ii}{\hbar}\right)^2 ∫_{-∞}^{t}\!\!\!\dd{t_2} ∫_{-∞}^{t_2}\!\!\!\dd{t_1} \tilde{\bm{H}}'(t_2) \tilde{\bm{H}}'(t_1) \bm{a}^{(0)}(t_1).
\end{equation}
这表明二阶微扰对应一个二次跃迁的过程,分别发生于$t_1,t_2$时刻(且$t_1<t_2<t$)。
由于二阶微扰涉及两次跃迁,因此其跃迁概率幅与$H'$的强度的平方成正比。
我们也可以具体写出式\eqref{eq:perturb_timedep_second_order_mat}的分量形式:
在Heisenberg绘景中,写作
\begin{equation}
    a_n^{(2)}(t) = \left(-\frac{\ii}{\hbar}\right)^2 ∑_{m_2,m_1} ∫_{-∞}^t\!\!\!\dd{t_2} ∫_{-∞}^{t_2}\!\!\!\dd{t_1} \tilde{H}'_{nm_2}(t_2) \tilde{H}'_{m_2m_1}(t_1) a_{m_1}^{(0)}(t_1);
\end{equation}
在Schrödinger绘景中,写作
\begin{equation}
    \label{eq:perturb_timedep_second_order_schrodinger}
    a_n^{(2)}(t) = \left(-\frac{\ii}{\hbar}\right)^2 ∑_{m_2,m_1} ∫_{-∞}^t\!\!\!\dd{t_2} ∫_{-∞}^{t_2}\!\!\!\dd{t_1} H'_{nm_2}(t_2) \ee^{\iiω_{nm_2}t_2} H'_{m_2m_1}(t_1) \ee^{\iiω_{m_2m_1}t_1} a_{m_1}^{(0)}(t_1).
\end{equation}

\begin{tcolorbox}[breakable, colframe=purple, colback=red!10, title={\textbf{高阶含时微扰的一般表式}}]
\it\small

高阶含时微扰可以由$λ$的对应阶次方程推导出来,对于第$k$阶,是
\begin{equation}
    \ii\hbar\dot{\bm{a}}^{(k)}(t) = \tilde{\bm{H}}'(t) \bm{a}^{(k-1)}(t),
\end{equation}
对应的解是
\begin{equation}
    \bm{a}^{(k)}(t) = -\frac{\ii}{\hbar}∫_{-∞}^t \tilde{\bm{H}}'(τ) \bm{a}^{(k-1)}(τ) \dd{τ}.
\end{equation}
反复迭代,即可得到第$k$阶微扰的表式
\begin{equation}
    \bm{a}^{(k)}(t) = \left(-\frac{\ii}{\hbar}\right)^k
    ∫_{-∞}^t\!\!\!\!\dd{t_k} ∫_{-∞}^{t_k}\!\!\!\!\dd{t_{k-1}} ∫_{-∞}^{t_{k-1}}\!\!\!\!\dd{t_{k-2}}⋯∫_{-∞}^{t_2}\!\!\!\!\dd{t_1} \tilde{\bm{H}}'(t_k) \tilde{\bm{H}}'(t_{k-1})⋯\tilde{\bm{H}}'(t_1) \bm{a}(t_1),
\end{equation}
这反映出,$k$阶微扰描述了一系列于$t_1,t_2,⋯,t_{k-1},t_k$时刻顺序发生的跃迁。

\end{tcolorbox}
