\documentclass[fontset=fandol, zihao=-4]{ctexart}
\usepackage[a4paper, scale=0.7]{geometry}
\usepackage{amsmath,amssymb,amsfonts}
\numberwithin{equation}{section}    % let the equation number displays in (1.1) style
\numberwithin{figure}{section}
\usepackage{unicode-math}
\setmathfont{STIX2Math}[Extension={.otf}, Path=./STIX2fonts/, Scale=1]
\setmainfont{STIX2Text}[Extension={.otf}, Path=./STIX2fonts/, UprightFont={*-Regular}, BoldFont={*-Bold}, ItalicFont={*-Italic}, BoldItalicFont={*-BoldItalic}]
\usepackage{graphicx}
\usepackage{xcolor}
\usepackage{hyperref}
\usepackage{physics}    % symbols related to physics
\usepackage{ulem}       % strikethrough \sout
\hypersetup{colorlinks=true,linkcolor=blue,citecolor=blue,urlcolor=magenta} % hyperref setup
% shortcuts
    % color
    \newcommand{\cyan}[1]{\textcolor{cyan}{#1}}
    % math style
    \renewcommand{\rm}[1]{\mathrm{#1}}  % roman, redefining original \rm
    \newcommand{\bm}[1]{\symbfit{#1}}   % bold+italic, using \symbfit instead of conventional \bm
    \newcommand{\br}[1]{\symbf{#1}}     % bold+roman
    % math
    \newcommand{\pd}{\partial}      % partial symbol
    % \newcommand{\dd}{\mathrm{d}}  % roman d (derivative & integral) [already implemented in the physics pkg]
    \newcommand{\ii}{\mathrm{i}}    % roman i (imag.)
    \newcommand{\ee}{\mathrm{e}}    % roman e (exp)
    % \newcommand{\abs}[1]{\lvert #1 \rvert}  % abs symbol [already implemented in the physics pkg]
    \renewcommand{\Re}{\mathcal{R}} % Re symbol
    \renewcommand{\Im}{\mathcal{I}} % Im symbol
    \newcommand{\del}{\br{\nabla}}  % bold nabla symbol
    % frequently-used notations
    \newcommand{\rr}{\bm{r}}
    \newcommand{\pp}{\bm{p}}

\begin{document}

\title{\LARGE \textbf{量子力学期末复习}\\ \textsc{Review Notes for Quantum Mechanics}}
\author{\Large \textsc{Mingyu Zhu}}

\setcounter{page}{0}
\maketitle
\thispagestyle{empty}
\tableofcontents
\pagebreak


\section*{前言}
\markboth{前言}{前言}
\addcontentsline{toc}{section}{前言}

\pagebreak

\section{量子力学基本原理}

% =================================================
\subsection{不确定性原理}

量子力学与经典力学之间的一大矛盾就在于\emph{不确定性原理}(uncertainty principle)。
在原子尺度上,经典力学的运动观念与实验的结论出现了巨大的鸿沟。
例如,在经典电动力学的框架下,围绕原子核做轨道运动的电子会释放电磁波从而最终落入原子核中;
电子衍射实验中,电子的行为呈现出波动特征(这种现象称为\emph{波粒二象性},wave-particle duality
\footnote{需要阐明的是,波粒二象性本身便可以视为不确定性原理的一项推论。}
)。

为了更准确地描述不确定性原理,我们需要明确\emph{测量}这一概念。
这里测量通常指的是一个服从经典力学的\emph{仪器}与被测量体系的相互作用,从而获得与体系有关的某个物理量的过程。
测量是理想的,这是说,我们在此不考虑仪器存在的噪声、误差等因素。

然而,不确定性原理表明,即便排除了仪器的噪声和干扰,我们所得的物理量测量结果也往往是不能用经典理论解释的。
例如,倘若我们反复测量一个低能电子的位置$\rr$,并通过逼近的手段试图求取电子的瞬时速度(即$\bm{v}=\lim_{\Delta t\rightarrow 0}\Delta\rr/\Delta t$),我们会发现所得的瞬时速度根本不收敛,
这意味着经典力学中的轨迹概念在量子力学中彻底失效了——在经典力学中,我们可以通过测量某一时刻系统的全部参数(广义坐标$q$与动量$p$),从而通过运动方程准确地预言系统之后的任何行为;而一个量子体系无法同时拥有确切的坐标和速度,因而也无法被精确预测。

不过,尽管我们无法准确预言量子体系的行为,但我们可以预言测量量子体系得到某一结果的概率(否则量子力学就要成为一门玄学了)。
这就是说,\emph{量子力学并不给出下一次测量的确切结果,而是给出测量结果的概率分布。}


% =================================================
\subsection{\texorpdfstring{量子态的描述\quad 态叠加原理\quad 统计诠释}{量子态的表示  态叠加原理  统计诠释}}

既然量子力学中没有了经典力学中的轨迹概念,经典力学的广义坐标、动量$(q,p)$显然不能用于描述量子体系,因此,我们有必要考虑如何完备地描述量子体系。
我们发现,当一个量子体系的某一组物理量都拥有确定的值之时,我们便唯一地确定了这个体系——测量这个体系的任何其他不含时的物理量所得的概率分布都是确定的。
因此,我们得到这样的结论:\emph{量子体系的描述依赖于选取的某一组力学量,以这组力学量处于定值的状态作为对量子态的完全描述。}
需要强调两点:
\begin{itemize}
    \item{力学量的选取并不是唯一的,可以有多种取法。}
    \item{为了完备描述我们的量子态,这一组力学量必须是“适定”“超定”的,否则无法达成完备性。}
    \item{这些力学量在经典力学中必须是完全非共轭的,例如,不能同时选取共轭的$\rr$和$\pp$。}
\end{itemize}
这一组力学量如何选取,有待后续讨论,但我们可以举出一些常见的例子:
\begin{itemize}
    \item{最为我们熟知的几种选取方法有坐标$\rr$、动量$\pp$或者能量(Hamiltonian量)$H$。}
    \item{如果读者学过原子物理,可以知道,对位于质子库仑场中的电子(即不考虑核自由度的氢原子体系),选取的一组力学量是能量$H$、角动量平方$L^2$、角动量$z$分量$L_z$与电子自旋$z$分量$s_z$,这一组力学量是“适定”的。}
    \item{描述光场,我们可以选取两种线偏振态$\leftrightarrow$与$\updownarrow$,也可以选取两种圆偏振态$\circlearrowleft$与$\circlearrowright$。}
\end{itemize}

确定了量子态的描述方法后,我们便可以继续思考如何描述一个普遍的量子态。
\emph{态叠加原理}揭示了这样一个事实:\emph{一个普遍的量子态,其可以表示为其他量子态的线性叠加。}

我们设在一个量子态中测量某一个力学量$q$可以得到确定的值$q_1$,我们将这个量子态记为$\psi_{q_1}$或$\psi(q_1)$
\footnote{一般来说,如果力学量$q$的可取值是离散的,可以将拥有确定取值$q_0$的量子态记为$\psi_{q_0}$,即将$q_0$记在下标;如果$q$的可取值是连续的,在表示对应的量子态时,可以将取值$q_0$写成函数变量的形式,即$\psi(q_0)$。};
对另一个态$\psi_{q_2}$进行$q$的测量可以获得$q_2$的取值,
这两个态可以按任意系数进行线性叠加形成一个新的量子态:
\begin{equation}
    \label{eq:intro_superposition_discrete}
    c_1 \psi_{q_1} + c_2 \psi_{q_2},
\end{equation}
或者写成向量形式:
\begin{equation}
    \begin{bmatrix}
        c1 \\ c2
    \end{bmatrix}
    \begin{bmatrix}
        \psi_{q_1} & \psi_{q_2}
    \end{bmatrix}.
\end{equation}
在这个态中,测量$q$的取值可能为$q_1$也可能为$q_2$,但\emph{其概率分布是确定的}。

对于离散变量$q$,一个相当好的例子就是描述任意偏振态的光子,选取$\{\leftrightarrow,\updownarrow\}$或$\{\circlearrowleft,\circlearrowright\}$作为$\{\psi_{q_1},\psi_{q_2}\}$,改变叠加系数$c_1, c_2$,即可描述任意偏振的光子。
这时我们说$\{\psi_{q_1},\psi_{q_2}\}$构成一组正交完备的\emph{基底}(basis)。

若$q$是一个连续变量,拥有连续的取值,那么我们很自然地发现,这种连续变量表示的量子态可以用一个关于$q$的函数来表达,即$\psi(q)$,而属于取值$q_0$的叠加系数,就是$\psi(q_0)$。
我们最熟悉的连续取值的力学量自然是坐标$\rr$了,这时,$\psi(\rr_0)$就表示位于坐标$\rr_0$处的这一量子态对应的叠加系数,而$\psi(\rr)$,就是我们熟知的\emph{波函数}。
对于动量$\pp$,相应的函数$\psi(\pp)$就称为\emph{动量波函数}。

在这里,我们强调:
\emph{给定的量子态是唯一确定的,而选取不同的力学量及其取值所对应的量子态所构成的基底描述这一量子态,得到的叠加系数是不同的。}
这正如同一个矢量在不同的坐标系中有着不同的表示系数,而这坐标系,正是我们为了描述这个量子态选取的一组基底,不同的基底就对应不同的\emph{表象}(representation)。
例如,使用波函数表示量子态,意味着我们选取了坐标表象;使用动量波函数,意味着我们选取了动量表象。
\emph{在研究量子力学问题时,明确所使用的表象尤为重要。}

显然,对式\eqref{eq:intro_superposition_discrete}所描述的量子态,测得力学量$q$取值为$q_1,q_2$的概率分布是仅仅由$c_1, c_2$确定的。



% =================================================
\subsection{\texorpdfstring{力学量的平均值\quad 力学量算符}{力学量的平均值  力学量算符}}


% 当然,也有一些处于特殊地位的物理量,在测量某一个给定的量子态之时,这些物理量的取值总是一个定值,这意味着这些物理量是体系的\emph{守恒量}。
% 对于守恒量,我们发现,测量的仪器往往只能给出一些特定的取值——
% 例如,研究氢原子光谱时,我们发现氢原子的能量(即Hamiltonian $H$)取值是离散的;放置了两片偏振片的偏振光实验中,探测器仅能通过探测到/未探测到光子告诉我们光子的偏振是平行/垂直于检偏器的透振方向。
% 之后我们将阐明守恒量在量子力学中的重要地位——它们是描述体系量子状态的重要依据。

% =================================================
\subsection{\texorpdfstring{量子态的演化\quad \schrodinger 方程}{量子态的演化  \schrodinger 方程}}



\pagebreak

\end{document}
