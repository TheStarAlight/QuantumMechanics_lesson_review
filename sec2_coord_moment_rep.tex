
\section{坐标表象与动量表象}
\label{sec:coord_moment_rep}

本章我们将在读者所熟知的坐标($\rr$)与动量($\pp$)表象下探讨量子力学基本原理的表现形式。
我们刻意将这一章与第\ref{sec:principles}章分开,是为了强调,坐标表象与动量表象仅仅是量子力学众多表象的其中两种。

% =================================================
\subsection{\texorpdfstring{坐标算符与动量算符\quad 正则对易关系}{坐标算符与动量算符  正则对易关系}}
\label{subsec:cmr_operators}

% ================================
\subsubsection{正则对易关系}

坐标表象与动量表象之间的联系是通过对易关系建立起来的。
\begin{tcolorbox}
\emph{坐标与动量这一对共轭变量的对易关系是}
\begin{equation}
    \label{eq:cmr_commutator}
    [x, p_x] = \ii\hbar,
\end{equation}
其中$\hbar=h/2\pi$是\emph{约化Planck常数}。
\end{tcolorbox}
通过这一对易关系,我们可以构建出坐标与动量算符在相应表象中的表示,并进一步导出它们的表象变换。

\begin{tcolorbox}
\textbf{注意}
\begin{itemize}
    \item{坐标与坐标、动量与动量之间均彼此对易,}
    \item{当推广至三维情形,不同方向的坐标与动量分量均彼此对易,}
\end{itemize}
即对于$i,j$取$x,y,z$时,有
\begin{equation}
    [q_i, q_j]=0,\quad [p_i, p_j]=0,\quad [q_i,p_j]=\ii\hbar\delta_{ij}.
\end{equation}
\end{tcolorbox}


% ================================
\subsubsection{坐标、动量算符的表示}

我们首先研究坐标表象。
坐标表象下,我们选取了坐标$\rr$的本征态作为描述任意量子态的基底,因此一个任意量子态$\ket{\psi}$应该可以表为一系列坐标本征态的叠加:
\begin{equation}
    \ket{\psi} = \int \dd\rr' f(\rr') \ket{\psi_{\rr'}}
\end{equation}
这里的$f(\rr)=\braket{\psi_{\rr}}{\psi}$正是坐标为$\rr$的坐标本征态的叠加系数,因此$f$表现为一个以坐标为自变量的复值函数,我们一般直接用$\psi$记号,称为\emph{(坐标)波函数(wavefunction)}。

\begin{tcolorbox}
在坐标表象下,坐标算符应当是对角的,且其对应本征值就是坐标值本身,因此其在坐标表象下的表示应为
\begin{equation}
    \hat{\rr}^{(\rr)} = \rr.
\end{equation}
一维动量$p_x$的算符表达式可以从式\eqref{eq:cmr_commutator}中得出:
\begin{equation}
    \hat{p}_x^{(\rr)} = -\ii\hbar\pd_x,
\end{equation}
可以验证,对于任意波函数$\psi(\rr)$,都有
\begin{equation}
    [x,\hat{p}_x] \psi(\rr) = -\ii\hbar [x\cdot\pd_x\psi(\rr) - \pd_x\cdot (x\psi(\rr))] = \ii\hbar \psi(\rr).
\end{equation}
对于三维动量$\pp=\{p_x,p_y,p_z\}$,有
\begin{equation}
    \hat{\pp}^{(\rr)} = -\ii\hbar
    \begin{bmatrix}
        \pd_x \\ \pd_y \\ \pd_z
    \end{bmatrix}
    = -\ii\hbar\del.
\end{equation}
\end{tcolorbox}

在动量表象中,一个量子态被表为一系列动量本征态的叠加,叠加系数称为动量波函数(momentum wavefunction),记为$\psi(\pp)=\braket{\psi_{\pp}}{\psi}$。
\begin{tcolorbox}
    在动量表象中,动量算符是对角的,为
    \begin{equation}
        \hat{\pp}^{(\pp)} = \pp;
    \end{equation}
    坐标算符的表示为
    \begin{equation}
        \hat{\rr}^{(\pp)} = \ii\hbar\del_{\pp}.
    \end{equation}

    \textbf{注意}\quad 坐标表象与动量表象下的算符关系不是简单地对换算符形式。
    动量算符在坐标表象下的表示中,微分算子的微分变量是坐标(显然,在坐标表象下,自变量只有坐标),而坐标算符在动量表象下的表示中,微分算子的微分变量是动量(因此特地予以标注)。
    此外,这两个算符形式相差一个负号,请予注意。
\end{tcolorbox}

\begin{tcolorbox}[breakable, colframe=blue, colback=blue!10, title={\textbf{含坐标、动量算符的对易子的计算}}]

    对易子有以下几个重要性质($f,g,h$为算符,$c$为常数):
    \begin{itemize}
        \item{$[f,g]=-[g,f]$\quad(交换取负)}
        \item{$[cf, g]=[f,cg]=c[f,g]$\quad (常数可提出)}
        \item{$[f+g, h]=[f,h]+[g,h]$\quad (算符加法分配)}
        \item{$[fg, h]=f[g,h]+[f,h]g$\quad (算符乘法分配\footnote{可以理解为在等号右侧的两项中:前者,$g$将左侧的$f$“推”至算符外边;后者,$f$将右侧的$g$“推”至算符外边。})}
    \end{itemize}

    在计算含坐标、动量算符时注意以下几个可能的技巧:
    \begin{itemize}
        \item{含$r^2, p^2$的力学量,考虑拆为$x^2+y^2+z^2$或$p_x^2+p_y^2+p_z^2$。}
        \item{含点乘的力学量,例如$\frac12(\rr\cdot\pp+\pp\cdot\rr)$,可考虑表为分量相乘形式:$\frac12(x p_x + p_x x + y p_y + p_y y + z p_z + p_z z)$。}
        \item{对于较为复杂的力学量,例如$1/r, 1/r^2$,考虑使用如下结论:
            \begin{equation}
                [\pp,F(\rr,\pp)] = -\ii\hbar\del_{\rr}F(\rr,\pp),\quad [\rr,F(\rr,\pp)] = \ii\hbar\del_{\pp}F(\rr,\pp).
            \end{equation}
        }
        \item{运算结果为矢量的对易子,可以考虑分别证明各个分量。}
    \end{itemize}

    此外,角动量也是经常出现于习题的力学量。
    角动量定义为$\bm{l}\coloneq\rr\times\pp$,其分量表示为
    \begin{equation}
        l_z = x p_y - y p_x,\quad l_x = y p_z - z p_y,\quad l_y = z p_x - x p_z.
    \end{equation}
    引入Levi-Civita符号$\eps_{ijk}$,其满足
    \begin{equation}
        \eps_{ijk} =
        \begin{cases}
            0,  & (i=j) \vee (j=k) \vee (i=k),\\
            1,  & \rm{\ for\ even\ permutation\ }(i,j,k),\\
            -1, & \rm{\ for\ odd\ permutation\ }(i,j,k).
        \end{cases}
    \end{equation}
    可以以之表示角动量与其他力学量的对易关系:
    \begin{equation}
        [l_i, q_j] = \eps_{ijk} \ii\hbar q_k, \quad
        [l_i, p_j] = \eps_{ijk} \ii\hbar p_k, \quad
        [l_i, l_j] = \eps_{ijk} \ii\hbar l_k.
    \end{equation}
    关于角动量,还可证明
    \begin{equation}
        \bm{l} \times \bm{l} = \ii\hbar\bm{l},
    \end{equation}
    \begin{equation}
        [l^2, l_i] = 0 \rm{\ for\ } i=x,y,z.
    \end{equation}
\end{tcolorbox}

\begin{tcolorbox}[breakable, colframe=purple, colback=red!10, title={\textbf{关于连续力学量的算符}}]
当我们讨论一个连续力学量时,态矢与算符该如何表示?
首先我们需要明确,对于离散变量,态矢所对应的表示向量是有一一对应的本征值的,即我们选取了这样的对应方式:
\begin{equation}
    \overbrace{
    \begin{bmatrix}
        q_1 \\ q_2 \\ \vdots \\ q_n
    \end{bmatrix}}^{\rm{quantity}}
    \Longleftrightarrow
    \overbrace{
    \begin{bmatrix}
        c_1 \\ c_2 \\ \vdots \\ c_n
    \end{bmatrix}}^{\rm{coefficient}}.
\end{equation}
现在,对于连续变量,事实上我们也可以考虑一个类似的对应关系:
\begin{equation}
    \overbrace{
    \begin{bmatrix}
        q=0.0 \cdots 00 \\ q=0.0 \cdots 01 \\ q=0.0 \cdots 02 \\ \vdots
    \end{bmatrix}}^{\rm{quantity}}
    \Longleftrightarrow
    \overbrace{
    \begin{bmatrix}
        \psi(0.0 \cdots 00) \\ \psi(0.0 \cdots 01) \\ \psi(0.0 \cdots 02) \\ \vdots
    \end{bmatrix}}^{\rm{coefficient}},
\end{equation}
我们看到,一个函数可以被看作一个无穷维的、稠密的向量。

如何理解态矢的内积\footnote{这里的内积,左矢需求共轭。}?
我们知道,两个离散向量$\{a_n\},\{b_n\}$的内积可以写作
\begin{equation}
    \bm{a}\cdot\bm{b} \coloneq \sum_n a_n^* b_n,
\end{equation}
类比之下,函数$f(q), g(q)$的内积自然由求和变为积分:
\begin{equation}
    f \cdot g \coloneq \int \dd q\ f^*(q) g(q).
\end{equation}

如何理解线性算符?
我们知道,离散变量的线性算符可以表为矩阵形式,其本质是对一个向量的线性操作,我们令$\bm{b}=Q\bm{a}$,那么:
\begin{equation}
    b_m = \sum_n Q_{mn} a_n,
\end{equation}
注意到,$Q_{mn}$贡献了从$a_n$到$b_m$的线性叠加。
那么对于连续变量,我们自然也可以写出这一算符的形式,令$g=Qf$,那么:
\begin{equation}
    \label{eq:cmr_operator_cont}
    g(q') = Q(q',q) f(q),
\end{equation}
$Q(q',q)$即贡献了从$f(q)$到$g(q')$的线性叠加。

那么求导算子$\partial_q$又如何理解?
如果我们将函数$f(q)$写成离散形式,令$q_{n+1} = q_n + \Delta q$,我们可以依中心差分写出导数的近似值:
\begin{equation}
    f'(q_n) \approx \frac{f(q_{n+1})-f(q_{n-1})}{2\Delta q},
\end{equation}
那么我们可以依此写出求导算子的矩阵形式:
\begin{equation}
    \partial_q \approx \frac{1}{2\Delta q}
    \begin{bmatrix}
        0  &  1 \\
        -1 &  0 &  1    \\
           & -1 &  0    &\ddots \\
           &    &\ddots &\ddots \\
    \end{bmatrix}.
\end{equation}
于是我们可以看到,其实求导也是一种线性运算,也可以近似写成矩阵的形式。
\end{tcolorbox}


% =================================================
\subsection{\texorpdfstring{坐标与动量的本征态\quad 表象变换}{坐标与动量的本征态  表象变换}}
\label{subsec:cmr_eigen_rep_trans}

% ================================
\subsubsection{坐标表象}

我们首先研究坐标、动量本征态在坐标表象下的表示。

\begin{tcolorbox}
在坐标表象下,坐标为$\rr_0$的本征态可以通过本征方程$\hat{\rr}\psi_{\rr_0}(\rr) = \rr\psi_{\rr_0}(\rr) = \rr_0 \psi_{\rr_0}(\rr)$求得,归一化的坐标本征态波函数为
\begin{equation}
    \psi_{\rr_0}(\rr) = \delta(\rr-\rr_0).
\end{equation}
而动量为$\pp_0$的动量本征态通过$\hat{\pp}\psi_{\pp_0}(\rr) = -\ii\hbar\del\psi_{\pp_0}(\rr) = \pp_0 \psi_{\pp_0}(\rr)$求得,其波函数为
\begin{equation}
    \psi_{\pp_0}(\rr) = \frac{1}{(2\pi\hbar)^{3/2}} \ee^{\ii\pp_0\cdot\rr/\hbar}.
\end{equation}
注意,这里的归一化系数的选取使得该态在动量空间归一化为$\delta$函数。
我们可以看到,\emph{动量本征态的坐标波函数表现为平面波的形式},其波长为$\lambda=2\pi\hbar/p_0=h/p_0$,这正是\emph{de Broglie关系}。
\end{tcolorbox}

求得坐标、动量本征态在同一表象下的表示后,我们就可以求解坐标、动量之间的表象变换关系,回忆式\eqref{eq:principles_rep_trans_mat}与\eqref{eq:cmr_operator_cont},我们写出$\rr\rightarrow\pp$表象变换的矩阵元
\begin{equation}
    S(\pp,\rr) = \braket{\psi_{\pp}}{\psi_{\rr}} = \int\dd\rr \psi_{\pp}^*(\rr) \psi_{\rr}(\rr) = \frac{1}{(2\pi\hbar)^{3/2}} \ee^{-\ii\pp\cdot\rr/\hbar},
\end{equation}
\begin{tcolorbox}
于是我们可以写出$\rr\rightarrow\pp$的表象变换\footnote{注意,这里使用了一个投影算符$\ket{\psi_{\rr}}\bra{\psi_{\rr}}$,其满足$\int\dd\rr\ket{\psi_{\rr}}\bra{\psi_{\rr}}=1$。}:
\begin{equation}
    \braket{\psi_{\pp}}{\psi}
    = \int\dd\rr \braket{\psi_{\pp}}{\psi_{\rr}}\braket{\psi_{\rr}}{\psi}
    = \int\dd\rr S(\pp,\rr)\psi(\rr)
    = \int\dd\rr \frac{\ee^{-\ii\pp\cdot\rr/\hbar}}{(2\pi\hbar)^{3/2}} \psi(\rr).
\end{equation}
我们发现,这与Fourier变换具有完全相同的形式。
\end{tcolorbox}

% ================================
\subsubsection{动量表象}

在动量表象下,情形也是类似的:
动量$\pp_0$的本征态为
\begin{equation}
    \psi_{\pp_0}(\pp) = \delta(\pp-\pp_0).
\end{equation}
坐标$\rr_0$的本征态为
\begin{equation}
    \psi_{\rr_0}(\pp) = \frac{1}{(2\pi\hbar)^{3/2}} \ee^{-\ii\rr_0\cdot\pp/\hbar}.
\end{equation}
而$\pp\rightarrow\rr$的表象变换矩阵$S^{-1}$,利用$S$的幺正性,可以写出其矩阵元
\begin{equation}
    S^{-1}(\rr,\pp) = S^\dag (\rr,\pp) = S^*(\pp,\rr)
    = \frac{1}{(2\pi\hbar)^{3/2}} \ee^{\ii\rr\cdot\pp/\hbar},
\end{equation}
于是表象变换为
\begin{equation}
    \braket{\psi_{\rr}}{\psi}
    = \int\dd\pp \braket{\psi_{\rr}}{\psi_{\pp}}\braket{\psi_{\pp}}{\psi}
    = \int\dd\pp S^{-1}(\rr,\pp)\psi(\pp)
    = \int\dd\pp \frac{\ee^{\ii\rr\cdot\pp/\hbar}}{(2\pi\hbar)^{3/2}} \psi(\pp).
\end{equation}

在进行表象变换时,牢记在心:原表象、目标表象分别为何,积分变量是原表象的变量!


% =================================================
\subsection{坐标表象下的\schrodinger 方程}
\label{subsec:cmr_schrodinger}

我们来考察一个处于保守势场$V(\rr)$中质量为$m$的粒子,其Hamiltonian量为
\begin{equation}
    H = \frac{p^2}{2m} + V(\rr).
\end{equation}
我们在坐标表象下研究这个问题,将动量算符的具体形式$\pp\rightarrow -\ii\hbar\del$代入\schrodinger 方程,可得
\begin{tcolorbox}
\emph{坐标表象下的\schrodinger 方程}:
\begin{equation}
    \ii\hbar\pd_t \psi(\rr,t) = H\psi(\rr,t) = \left[-\frac{\hbar^2}{2m}\nabla^2 + V(\rr)\right]\psi(\rr,t).
\end{equation}
\end{tcolorbox}
这一方程将在之后的问题中频繁出现。
我们看到,这一方程形似波动方程,这也是\schrodinger 所提出的量子力学框架被称为“波动力学”的原因——\schrodinger 方程揭示了物质波的演化规律。

