
\section{中心力场}
\label{sec:central_field}

中心力场是一种势能仅与$r$有关的力场,势能可以写作$V(r)$。
写出在中心势$V(r)$中质量为$m$的粒子的Hamiltonian算符
\begin{equation}
    H = \frac{p^2}{2m} + V(r).
\end{equation}
可以证明角动量是守恒量:$[\bm{l},H]=\bm{0}$,这与经典力学的结论是一致的。

在坐标表象、球坐标下求解中心力场问题一般较为方便。
写出Hamiltonian算符在坐标表象、球坐标中的形式,定态\schrodinger 方程可以分离出径向和角向部分,定态波函数解也可以被表为径向和角向分离变量的形式:
\begin{equation}
    \psi(r,\theta,\phi) = R(r) \rm{Y}_{lm}(\theta,\phi),\quad l=0,1,2,\cdots,\ m=-l,-l+1,\cdots,l,
\end{equation}
其中$Y_{lm}$是球谐函数,是$l^2$与$l_z$的共同本征函数;
径向函数$R(r)$满足方程
\begin{equation}
    R_l'' + \frac{2}{r} R_l' + \left[\frac{2m}{\hbar^2}(E-V(r))-\frac{l(l+1)}{r^2}\right] R_l = 0,
\end{equation}
可见在中心力场中,粒子波函数的差别仅在于径向部分。
选取一定的边界条件求解径向方程,对于束缚态($E<\lim_{r\rightarrow\infty} V(r)$),可得到径向量子数$n_r$及其对应的能量本征值$E_{n_r l}$。
径向方程中不出现$m$,说明$E$与$m$无关,即中心力场中每个能级都至少有$m$个\emph{简并(degenerate)}的量子态。

如此,我们选用了$\{H, l^2, l_z\}$作为描述中心力场量子态的力学量集,并用$n_r, l, m$(径向量子数、角量子数\footnote{$l=0,1,2,3$分别对应字母s,p,d,f,而之后沿字母表顺延。}与磁量子数)三个量子数区分这些量子态。
这一力学量集足以区分各个简并的量子态,而且是“适定”的,因此称为\emph{力学量完全集}。

% =================================================
\subsection{三维各向同性球谐振子}
\label{subsec:cf_osc}

我们考虑一个质量为$m$、位于三维各向同性谐振子势
\begin{equation}
    V(r) = \frac12 m\omega^2 r^2
\end{equation}
中的粒子。

求解径向方程,利用无穷远处$R(r)\rightarrow 0$的边界条件,我们得到了能量本征值
\begin{equation}
    E = \left( 2n_r+l+\frac32 \right)\hbar\omega,\quad n_r,l=0,1,2,\cdots
\end{equation}
取$N=2n_r+l$,能量本征值写为
\begin{equation}
    E_N = (N+\frac32) \hbar\omega,\quad N=0,1,2,\cdots.
\end{equation}
可以发现,与一维谐振子相同,三维各向同性球谐振子的能级是均匀分布的,间距为$\hbar\omega$,不过零点能为$\frac32\hbar\omega$。

可以求解出对于给定的能级$E_N$,有多少可能的量子态(用$n_r,l,m$区分)共享同一个能级,即简并度是多少。
我们知道,对于给定的$N$,可能的$(n_r,l)$组合有
\begin{equation}
    (0,N),\ (1,N-2),\ (2,N-4)\cdots\cdots
    \begin{cases}
        ((N-1)/2, 1),   & \rm{for\ odd\ }N, \\
        (N/2, 0)        & \rm{for\ even\ }N, \\
    \end{cases}
\end{equation}
对于每一组$(n_r,l)$,都有$2l+1$个不同的$m$的量子态,因此简并度
\begin{equation}
    d_N =
    \begin{cases}
        \sum_{l=1,3,\cdots,N} = \frac12 (N+1)(N+2), & \rm{for\ odd\ }N, \\
        \sum_{l=0,2,\cdots,N} = \frac12 (N+1)(N+2), & \rm{for\ even\ }N. \\
    \end{cases}
\end{equation}
可见,对于奇数或者偶数的$N$,简并度均为
\begin{equation}
    d_N = \frac12 (N+1)(N+2).
\end{equation}

我们看到,三维各向同性球谐振子场具有比一般中心力场更高的简并度,这与其径向势场有关,更高的“对称性”导致了更多的简并态。

% =================================================
\subsection{氢原子}
\label{subsec:cf_hydrogen}

氢原子中有一个质子和电子,故其实际上是一个两体问题。
可以通过引入质心坐标和相对坐标的方法,将两体问题化为单体问题。
这样,仍然可以沿用之前的中心力场的处理方法,但质量需理解为约化质量,能量理解为相对运动能量。

在自然单位制$\hbar=m_e=\abs{e}=1$下,Coulomb势场的能量为
\begin{equation}
    V(r) = -\frac{1}{r}.
\end{equation}

由径向方程的边界条件,给出径向量子数$n_r=0,1,2,\cdots$,本征能量为
\begin{equation}
    E_n = -\frac{1}{2(n_r+l_1)^2},
\end{equation}
取主量子数$n:=n_r+l_1+1,\ n=1,2,\cdots$,可得
\begin{equation}
    E_n = -\frac{1}{2n^2}.
\end{equation}

能量的自然单位为$me^4/\hbar^2 = 1\ \rm{Hartree} \approx 27.211\ \rm{eV}$,
长度的自然单位是$a_0 = \hbar^2/me^2 \approx 0.529\ \rm{Å}$,为一个Bohr半径。
如此,本征能量表为
\begin{equation}
    E_n = -\frac{e^2}{a_0}\frac{1}{2n^2},
\end{equation}
其中基态的能量为$E_1 = -\frac{e^2}{a_0} \approx -13.6\ \rm{eV}$。

我们用$(n,l,m)$来区分氢原子的各个量子态。
\begin{itemize}
    \item 角动量平方的期望值$\expval{\LL^2}=l(l+1)\hbar^2$;
    \item 角动量$z$分量的期望值$\expval{L_z}=m\hbar$。
\end{itemize}

特别地,我们写出基态的波函数:
\begin{equation}
    \psi_{100}(r) = \frac{1}{\sqrt{\pi a_0^3}}\ee^{-r/a_0}.
\end{equation}
读者应牢记。

最后还有一个计算技巧。
在计算$r^n$的期望值时,可以采用如下的策略:
\begin{equation}
\begin{aligned}
    \expval{r^n}
    &= \frac{\int r^2 \dd r \dd\Omega \ Y_{lm}^*(\theta,\phi) r^n Y_{lm}(\theta,\phi) R_{nl}^2(r)}{\int r^2 \dd r \dd\Omega \ Y_{lm}^*(\theta,\phi) Y_{lm}(\theta,\phi) R_{nl}^2(r)} \\
    &= \frac{\int\dd r \ r^{n+2} R_{nl}^2(r)}{\int\dd r \ r^2 R_{nl}^2(r)},
\end{aligned}
\end{equation}
其中$R_{nl}(r)$是多项式$P(r)$与$\ee^{-r/na_0}$的乘积,因此问题最后可以归结为求解一个重要积分:
\begin{equation}
    \int_0^\infty \dd t \ t^{N} \ee^{-t} = \Gamma(N+1) = N!,
\end{equation}
这一公式需牢记。

\begin{tcolorbox}[breakable, title={\textbf{例题:氢原子能级叠加态}}]
    \paragraph{题目}
    氢原子在$t=0$时刻处于量子态($\ket{nlm}$表示量子数$n,l,m$的本征态)
    \begin{equation}
        \ket{\psi(0)} = \frac12\ket{210} - \frac{1}{\sqrt{2}}\ket{310} - \frac{1}{\sqrt{2}}\ket{21-1},
    \end{equation}
    求$H, L^2, L_z$三个力学量:
    \begin{enumerate}
        \item{在$t=0$时刻的测值概率分布及平均值;}
        \item{任意$t$时刻的测值概率分布及平均值。}
    \end{enumerate}

    \paragraph{解答}

    注意:这个量子态\emph{尚未归一化},我们先进行归一化:
    \begin{equation}
        \ket{\psi(0)} = \frac{1}{\sqrt{5}}\ket{210} - \sqrt{\frac25}\ket{310} - \sqrt{\frac25}\ket{21-1}.
    \end{equation}

    首先明确各个力学量的量子数与本征值的关系:
    \begin{itemize}
        \item{能量:我们不妨设基态($n=1$)时的能量为$E_0=-13.6\ \rm{eV}$,则主量子数为$n$的量子态的本征能量为$E_0/n^2$;}
        \item{角动量平方:角量子数为$l$的本征态对应的角动量平方大小为$l(l+1)\hbar^2$;}
        \item{角动量$z$分量:磁量子数为$m$的本征态对应的角动量$z$分量为$m\hbar$。}
    \end{itemize}

    根据统计诠释,测得$n=2$即$E_n=E_0/4$的概率由$\ket{210}$和$\ket{21-1}$贡献:
    \begin{equation}
        P_{E_2} = \abs{1/\sqrt{5}}^2 + \abs{-\sqrt{2/5}}^2 = 3/5,
    \end{equation}
    测得$n=3$即$E_n=E_0/9$的概率由$\ket{310}$贡献:
    \begin{equation}
        P_{E_3} = \abs{-\sqrt{2/5}}^2 = 2/5.
    \end{equation}
    能量平均值为
    \begin{equation}
        \expval{H} = \frac35 E_2 + \frac25 E_3 = \frac{7}{36} E_0.
    \end{equation}

    依此可得
    \begin{equation}
    \begin{aligned}
        P_{l=1} &= 1, \\
        \expval{L^2} &= 2\hbar^2; \\
        P_{m=0} &= 3/5, \\
        P_{m=-1} &= 2/5, \\
        \expval{L_z} &= \frac35 \cdot 0\hbar + \frac25 \cdot -\hbar = -\frac25\hbar.
    \end{aligned}
    \end{equation}

    对于含时情况,我们写出对应的时间相位
    \begin{equation}
        \ket{\psi(0)} = \left(\frac{1}{\sqrt{5}}\ket{210} - \sqrt{\frac25}\ket{21-1}\right) \ee^{-\ii E_0 t/4\hbar} - \sqrt{\frac25}\ket{310} \ee^{-\ii E_0 t/9\hbar}.
    \end{equation}
    在按前一题的方法计算测值概率时,只要将对应的系数乘以相应的量子相位即可。
    事实上,我们要注意到,在氢原子体系中,$H, L^2, L_z$三个力学量均与Hamiltonian量对易,因此是守恒量,因此无论系统是否处于单个能量本征态,其力学量测值概率分布均不随时间而变。
    因此$t>0$时刻的测值分布与平均值与$t=0$时完全一样。
    \hfill $\square$

\end{tcolorbox}

\begin{tcolorbox}[breakable, title={\textbf{例题}}]
    \paragraph{题目} \textit{(2025年考研·三)}
    质量为$m$的粒子在有心势场$V(r)=kr$中运动。
    以$\psi(r)\sim\ee^{-\lambda r}$为试探解,$\lambda$为变分参数,试用变分法估算体系的基态能量。

    \paragraph{解答}

    首先,试探解未归一化,因此第一步要写出变分参数$\lambda$对应的归一化试探解:
    \begin{equation}
        \psi(r) = N_\lambda \ee^{-\lambda r},
    \end{equation}
    归一化条件为
    \begin{equation}
    \begin{aligned}
        1 &= \int_0^\infty 4\pi r^2 \dd r \abs{\psi(r)}^2 \\
        &= 4\pi N_\lambda^2 \int_0^\infty r^2 \ee^{-2\lambda r} \dd r \\
        &= 4\pi N_\lambda^2 (2\lambda)^{-3} \int_0^\infty \rho^2 \ee^{-\rho} \dd\rho \\
        &= 4\pi N_\lambda^2 (2\lambda)^{-3} \cdot 2!,
    \end{aligned}
    \end{equation}
    可得归一化常数
    \begin{equation}
        N_\lambda = \sqrt{\frac{\lambda^3}{\pi}}.
    \end{equation}

    在坐标表象下计算试探解的能量期望值$\expval{H}_\lambda$,分别计算势能$\expval{V}$与动能$\expval{T}$:
    \begin{equation}
    \begin{aligned}
        \expval{V}_\lambda
        &= \int_0^\infty 4\pi r^2 \dd r \abs{\psi(r)}^2 V(r) \\
        &= 4\pi k N_\lambda^2 \int_0^\infty r^3 \ee^{-2\lambda r} \dd r \\
        &= 4\pi k N_\lambda^2 (2\lambda)^{-4} \int_0^\infty \rho^3 \ee^{-\rho} \dd\rho \\
        &= 4\pi k N_\lambda^2 (2\lambda)^{-4} \cdot 3! \\
        &= \frac{3k}{2\lambda},
    \end{aligned}
    \end{equation}
    \begin{equation}
    \begin{aligned}
        \expval{T}_\lambda
        &= \int_0^\infty 4\pi r^2 \dd r \psi(r) \cdot \left(-\frac{\hbar^2}{2m}\underbrace{\nabla^2 \psi(r)}_{\pd^2_{rr}\psi(r)}\right) \\
        &= \int_0^\infty 4\pi r^2 \dd r \psi(r) \cdot \left(\frac{\hbar^2}{2m}\lambda^2 \psi(r)\right) \\
        &= \frac{\hbar^2}{2m}\lambda^2,
    \end{aligned}
    \end{equation}
    于是参数$\lambda$对应的能量期望值是
    \begin{equation}
        \expval{H}_\lambda = \frac{3k}{2\lambda} + \frac{\hbar^2}{2m}\lambda^2,
    \end{equation}
    $\expval{H}$关于$\lambda$函数在$\pd_\lambda \expval{H} = 0$,对应
    \begin{equation}
        \lambda_0 = \sqrt[3]{\frac{2m}{3\hbar^2 k}}
    \end{equation}
    时取得极小值,为
    \begin{equation}
        E_0 = \frac94\sqrt[3]{\frac{2\hbar^2 k^2}{3m}}.
    \end{equation}
\end{tcolorbox}