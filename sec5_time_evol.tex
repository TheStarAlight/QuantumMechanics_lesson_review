\section{时间演化}
\label{sec:time_evol}

在\ref{subsec:principles_time_evolution}节中,我们已经讨论了一个量子系统随时间演化的规律,
其中Schrödinger方程描述了量子态随时间的变化规律,而Heisenberg方程则描述了力学量(平均值)的演化规律。
事实上,量子态是无法直接观测的,我们所能直接观察的物理实在也仅仅是系统的可观测力学量。
从这个角度来说,Heisenberg方程显得更加自然并具有基础性。

% =================================================
\subsection{时间演化算符}

我们假定一个量子系统在零时刻与$t$时刻的量子态分别为$\ket{\psi(0)}$和$\ket{\psi(t)}$。
根据Schrödinger方程
\begin{equation}
    \ii\hbar\pd_t\ket{\psi} = H\ket{\psi},
\end{equation}
我们可以得到Schrödinger方程的一个形式解:
\begin{equation}
    \ket{\psi(t)} = \exp\left(-\ii \int_0^t H \dd\tau /\hbar\right)\ket{\psi(0)},
\end{equation}
若$H$不含时,则为
\begin{equation}
    \ket{\psi(t)} = \exp(-\ii H t/\hbar)\ket{\psi(0)}.
\end{equation}

我们看到,这个特殊的算符
\begin{equation}
    U(t,0) := \exp\left(-\ii \int_0^t H \dd\tau /\hbar\right)
\end{equation}
将两个时刻的量子态联系起来,称为\emph{时间演化算符(time-evolution operator)}。
根据我们的物理直觉,这个算符显然具有\emph{幺正性(unitary)},即其是保留内积的变换,这是说
\begin{equation}
    \innerproduct{\psi(t)}{\psi(t)} = \mel{\psi(0)}{U^\dag U}{\psi(0)} = \mel{\psi(t)}{I}{\psi(0)} = \innerproduct{\psi(0)}{\psi(0)}.
\end{equation}
时间演化算符在形式上非常简单,然而在非对角的表象中,依然很难求出具体的形式。
\footnote{初学者可能对``算符的指数''存在误会。形式上,算符的指数可以直接按照指数函数的定义来理解,即$\exp A := \sum_{n=0}^\infty A^n/n!$。
    初学者可能认为``算符的指数的矩阵元''应该等于``原算符对应矩阵元的指数'',即$\{\exp A\}_{ij} = \exp(A_{ij})$,
    然而这是一种错误的观点(仅对``对角''算符成立),因为算符的指数是一个整体,它对整个算符进行指数运算,而不是对算符的每个元素分别进行指数运算。
}

在量子力学中,可观测力学量$F$的期望值是我们仅能直接观察的物理实在:
\begin{equation}
    \expval{F} = \mel{\psi(0)}{U^\dag F U}{\psi(0)},
\end{equation}
这个式子有两种不同的理解方式:
\begin{itemize}
    \item 第一种理解是认定力学量算符$F$保持不变,而量子态$\ket{\psi(t)}=U\ket{\psi(0)}$随时间演化,这就是说:
        \begin{equation}
            \expval{F} = \underbrace{\bra{\psi(0)}U^\dag}_{\bra{\psi(t)}} F \underbrace{U\ket{\psi(0)}}_{\ket{\psi(t)}},
        \end{equation}
        这种表述方式称为Schrödinger绘景(picture)。
    \item 第二种理解是认为力学量算符$F(t)=U^\dag F(0) U$随时间流易而演化,而量子态$\ket{\psi}$保持不变:
        \begin{equation}
            \expval{F} = \mel{\psi}{\underbrace{U^\dag F(0) U}_{F(t)}}{\psi}
        \end{equation}
        这种表述方式称为Heisenberg绘景。
\end{itemize}
两种绘景殊途同归,是完全等价的,在不同的问题中可以选择更适合的绘景进行分析。


% =================================================
\subsection{Virial定理}

我们研究一个位于势场$V(\rr)$中的粒子,其Hamiltonian:
\begin{equation}
    H = \frac{\pp^2}{2m} + V(\rr).
\end{equation}
对于定态,考察$\rr\cdot\pp$的期望值随时间的演化:
\begin{equation}
\begin{aligned}
    \frac{\dd}{\dd t}\expval{\rr\cdot\pp}
    &= \frac{1}{\ii\hbar} \expval{[\rr\cdot\pp, H]} \\
    &= \frac{1}{\ii\hbar} \expval{\rr\cdot[\pp,H] + [\rr,H]\cdot\pp}\\
    &= \frac{1}{\ii\hbar} \expval{-\ii\hbar\rr\cdot\nabla V(\rr) + \ii\hbar\pp^2/m} \\
    &= -\expval{\rr\cdot\nabla V(\rr)} + 2\expval{T},
\end{aligned}
\end{equation}
其中$T=\pp^2/2m$是动能。
对于定态,$\rr\cdot\pp$的期望值是一个常数,因此其时间导数为零,于是有
\begin{equation}
    2\expval{T} = \expval{\rr\cdot\nabla V(\rr)}.
\end{equation}
这就是\emph{Virial定理}。

这一定理对于特殊的势能函数具有特别的意义。
例如,对于$n$次齐次势能函数,有$\rr\cdot\nabla V(\rr) = nV(\rr)$,那么就有
\begin{equation}
    2\expval{T} = nV(\rr).
\end{equation}


% =================================================
\subsection{Ehrenfest定理}

对于一个量子波包,我们好奇的是,其坐标与动量随时间的变化率是否有与经典力学类似的形式?
我们设Hamiltonian为$H=\pp^2/2m + V(\rr)$,可以写出
\begin{equation}
\begin{aligned}
    \frac{\dd}{\dd t}\rr &= \frac{1}{\ii\hbar} \expval{[\rr, H]} = \frac{\expval{\pp}}{m}, \\
    \frac{\dd}{\dd t}\pp &= \frac{1}{\ii\hbar} \expval{[\pp, H]} = \expval{-\nabla V(\rr)} = \expval{\bm{F}(\rr)},
\end{aligned}
\end{equation}
可见这与经典力学Newton方程的形式非常相似,这就是\emph{Ehrenfest定理},其表明,量子波包的坐标与动量期望值的时间演化遵循类似经典力学的规律。


% =================================================
\subsection{Feynman-Hellmann定理}

Feynman-Hellmann定理指出,对于一个含参$\lambda$的Hamiltonian $H(\lambda)$,设$E_n$为其本征态$\ket{\psi_n}$的本征能量,则有
\begin{equation}
    \frac{\pd E_n}{\pd \lambda} = \mel{\psi_n}{\frac{\pd H}{\pd \lambda}}{\psi_n}.
\end{equation}

证明如下:
$E_n$为$H$的本征态$\ket{\psi_n}$的本征能量,因此有
\begin{equation}
    H\ket{\psi_n} = E_n \ket{\psi_n}.
\end{equation}
移项并对$\lambda$求导,得
\begin{equation}
    \left( \pd_\lambda H - \pd_\lambda E_n \right) \ket{\psi_n} + (H-E_n) \pd_\lambda \ket{\psi_n} = 0.
\end{equation}
左乘$\bra{\psi_n}$,并注意到$(H-E_n)\ket{\psi_n}=0$:
\begin{equation}
    \mel{\psi_n}{\pd_\lambda H - \pd_\lambda E_n}{\psi_n} + \underbrace{\bra{\psi_n} (H-E_n)}_{0} \pd_\lambda \ket{\psi_n} = 0.
\end{equation}
移项即得F-H定理,证毕。
