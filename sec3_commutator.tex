
\section{算符对易关系}
\label{sec:commutator}

在实际应试中,计算算符对易子$[A,B]=AB-BA$是一项十分重要的能力,
一些重要的结论亦需通过计算算符对易关系来证明。
本章我们将讨论计算算符对易子的思路与技巧。

% =================================================
\subsection{常用规则}
\label{subsec:commutator_rules}

计算对易子有几个常用的规则:
\begin{itemize}
    \item 交换取负:$[A,B] = -[B,A]$;
    \item ``加法分配律'':$[\alpha A+\beta B,C] = [\alpha A,C]+[\beta B,C]$;
    \item ``乘法分配律''(前推前,后推后):
    \begin{equation}
        [A, \blue{B}\red{C}] = [A,\blue{B}] \red{C} + \blue{B}[A,\red{C}].
    \end{equation}
\end{itemize}

我们还常遇到一些矢量算符。

若对易子本身为矢量,可以考虑将其分解为各个方向的分量。
例如,要证$\LL\times\pp+\pp\times\LL = 2\ii\hbar\pp$,分别证明其三个分量成立即可,
以$x$方向为例:$(\LL\times\pp+\pp\times\LL)_x = [p_y,L_z]+[L_y,p_z] = 2\ii\hbar p_x$。

若对易子中的算符是两个矢量的内积或外积,``乘法分配律''也是适用的:
\begin{equation}
    \begin{aligned}
        \relax
        [A, \blue{\bm{B}}\cdot\red{\bm{C}}] &= [A,\blue{\bm{B}}]\cdot\red{\bm{C}} + \blue{\bm{B}}\cdot[A,\red{\bm{C}}], \\
        [A, \blue{\bm{B}}\times\red{\bm{C}}] &= [A,\blue{\bm{B}}]\times\red{\bm{C}} + \blue{\bm{B}}\times[A,\red{\bm{C}}].
\end{aligned}
\end{equation}
内积也可以考虑拆为分量形式,例如:$\rr^2=x^2+y^2+z^2$。

对于外积,敬请牢记定义式:
\begin{equation}
    \bm{\blue{A}}\times\bm{\red{B}} =
    \begin{vmatrix}
        \hat{\bm{x}} & \hat{\bm{y}} & \hat{\bm{z}} \\
        \blue{A_x} & \blue{A_y} & \blue{A_z} \\
        \red{B_x} & \red{B_y} & \red{B_z}
    \end{vmatrix}
    = \hat{\bm{x}}\begin{vmatrix}\blue{A_y} & \blue{A_z} \\ \red{B_y} & \red{B_z}\end{vmatrix}
     -\hat{\bm{y}}\begin{vmatrix}\blue{A_x} & \blue{A_z} \\ \red{B_x} & \red{B_z}\end{vmatrix}
     +\hat{\bm{z}}\begin{vmatrix}\blue{A_x} & \blue{A_y} \\ \red{B_x} & \red{B_y}\end{vmatrix}.
\end{equation}


% =================================================
\subsection{坐标、动量与角动量的对易子}
\label{subsec:commutator_rpL}

计算坐标、动量与角动量的对易子,从最基本的坐标—动量对易关系
\begin{equation}
    [x, p] = i \hbar
\end{equation}
开始,并结合\ref{subsec:commutator_rules}节所述的对易子计算规则,即可得解。

当我们涉及的坐标与动量为三维时,需要注意的是,不同维度的坐标、动量之间彼此无条件对易,因此:
\begin{equation}
    [r_i, p_j] = i \hbar \delta_{ij}.
\end{equation}

当对易子涉及角动量时,问题变得更加复杂。
首先,请牢记角动量的定义及其分量表达式:
\begin{equation}
    \LL = \rr\times\pp =
    \begin{vmatrix}
        \hat{\bm{x}} & \hat{\bm{y}} & \hat{\bm{z}} \\
        x & y & z \\
        p_x & p_y & p_z \\
    \end{vmatrix}
    = \underbrace{(y p_z - z p_y)}_{L_x} \hat{\bm{x}} + \underbrace{(z p_x - x p_z)}_{L_y} \hat{\bm{y}} + \underbrace{(x p_y - y p_x)}_{L_z} \hat{\bm{z}}.
\end{equation}

接下来是三个角动量与坐标、动量与角动量算符分量对易子的常用二级结论:
\begin{equation}
    \begin{aligned}
        \relax
        [L_i, r_j] &= \ii\hbar\epsilon_{ijk} r_k, \\
        [L_i, p_j] &= \ii\hbar\epsilon_{ijk} p_k, \\
        [L_i, L_j] &= \ii\hbar\epsilon_{ijk} L_k,
    \end{aligned}
\end{equation}
这里$\epsilon_{ijk}$是Levi-Civita符号,仅在$i\ne j\ne k$时不为零,在$ijk$为偶排列时为$1$,奇排列时为$-1$
\footnote{判断是否为偶排列,可以通过判断需要对调几次元素才能回到正序排列$123$,对调次数为奇数就是奇排列,对调次数为偶数就是偶排列,这是利用了对调一次元素将改变排列的奇偶性的性质。}。

还有两个比较重要的二级结论:
\begin{equation}
    \LL\times\LL = \ii\hbar\LL,
\end{equation}
\begin{equation}
    [\LL^2, L_i] = 0.
\end{equation}


% =================================================
\subsection{``幂次''公式}

有时对易子$[A,B]$中的算符并不能简单地表为$A$和$B$的乘幂,例如,\ref{subsec:commutator_rpL}节中所述结论面对形如$1/r$的算符束手无策。
而``幂次''公式及其推论则提供了一种手段来处理这种情况。

我们设$[A, B] = C$,且$C$与$A$与$B$均对易。
易证
\begin{equation}
\begin{aligned}
    \relax
    [A,B^n]
    &= [A,\blue{B}\cdot \red{B^{n-1}}] \\
    &= \blue{B} [A, \red{B^{n-1}}] + \underbrace{[A, \blue{B}]}_C \red{B^{n-1}} \\
    &= B^2 [A, B^{n-2}] + 2C B^{n-1} \\
    &= \cdots \\
    &= n C B^{n-1}.
\end{aligned}
\end{equation}

这一公式可以进一步推广至$[A,f(B)]$的情形。
我们假定$f(B)$是解析函数,于是
\begin{equation}
\begin{aligned}
    \relax
    [A,f(B)]
    &= \left[A, \sum_n \frac{f_n}{n!} B^n\right] \\
    &= \sum_n \frac{f_n}{n!} [A, B^n] \\
    &= C \sum_n \frac{f_n}{(n-1)!} B^{n-1} \\
    &= C \frac{\pd f(B)}{\pd B}.
\end{aligned}
\end{equation}

对于$A,B$是坐标与动量(的函数)这一特殊情形,有
\begin{equation}
\begin{aligned}
    \relax
    [\rr, f(\rr,\pp)] = \ii\hbar\nabla_{\pp} f(\rr,\pp), \\
    [\pp, f(\rr,\pp)] = -\ii\hbar\nabla_{\rr} f(\rr,\pp).
\end{aligned}
\end{equation}
取一例:
\begin{equation}
    [\pp, \frac{1}{r}] = [p_x \hat{\bm{x}} + p_y \hat{\bm{y}} + p_z \hat{\bm{z}}, \frac{1}{r}] = -\ii\hbar \nabla \frac{1}{r} = \ii\hbar \frac{\hat{\bm{r}}}{r^2}.
\end{equation}
这里的$\nabla_{\rr}, \nabla_{\pp}$应当理解为\emph{``对算符求导数''}。
虽然看起来与$\rr,\pp$算符在彼此的表象中的形式
\begin{equation}
    \hat{\rr}^{(\pp)}\equiv\ii\hbar\nabla_{\pp},
    \quad \hat{\pp}^{(\rr)}\equiv-\ii\hbar\nabla_{\rr}
\end{equation}
完全一样,但本质完全不同。
请牢记:\emph{对易子的计算无关表象!}
因此,在计算对易子时,请勿直接将任何算符替换为其在另一个表象中的形式。


\begin{tcolorbox}[breakable, title={\textbf{例题:两个对易子的计算}}]
    \it\small
    \fbox{ \textbf{题目} }
    计算如下几个对易子:
    \begin{enumerate}
        \item
            \begin{equation}
                [\rr,H], [\pp,H]
            \end{equation}
            其中$H=\pp^2/2m + V(\rr)$;
        \item
        \begin{equation}
                [p^2,\frac{1}{r}].
            \end{equation}
    \end{enumerate}

    \fbox{ \textbf{解答} }

    1. 利用结论
    \begin{equation}
        [\pp,F(\rr,\pp)] = -\ii\hbar\del_{\rr}F(\rr,\pp),\quad [\rr,F(\rr,\pp)] = \ii\hbar\del_{\pp}F(\rr,\pp),
    \end{equation}
    可得
    \begin{equation}
    \begin{aligned}\relax
        [\rr, H]
        &= \ii\hbar\del_{\pp}\left[\pp^2/2m + V(\rr)\right] \\
        &= \ii\hbar\del_{\pp}\left[\pp^2/2m\right] \\
        &= \ii\hbar\pp/m;
    \end{aligned}
    \end{equation}
    \begin{equation}
    \begin{aligned}\relax
        [\pp, H]
        &= -\ii\hbar\del_{\rr}\left[\pp^2/2m + V(\rr)\right] \\
        &= -\ii\hbar\del V(\rr).
    \end{aligned}
    \end{equation}
    \hfill$\square$

    2. 考虑按算符乘积拆分成
    \begin{equation}
        [p^2,\frac{1}{r}] = \pp\cdot[\pp,\frac{1}{r}] + [\pp,\frac{1}{r}]\cdot\pp.
    \end{equation}
    接下来先计算$[\pp,\frac{1}{r}]$:
    \begin{equation}
        [\pp,\frac{1}{r}] = -\ii\hbar\del\frac{1}{r} = \ii\hbar\frac{\rr}{r^3}.
    \end{equation}
    代入即得
    \begin{equation}
        \pp\cdot[\pp,\frac{1}{r}] + [\pp,\frac{1}{r}]\cdot\pp = \ii\hbar\left( \frac{\rr}{r^3}\cdot\pp+\pp\cdot\frac{\rr}{r^3} \right),
    \end{equation}
    这一表达式可以继续化简,但\emph{不能直接代入动量算符在坐标表象中的表达式} $\hat{\pp}^{(\rr)}=-\ii\hbar\del$\footnote{对易子的计算无关表象,因此不能直接代入,而要使用之前的结论,构造对易子来求解。}:
    \begin{equation}
    \begin{aligned}
        &\quad\ \ii\hbar\left( \frac{\rr}{r^3}\cdot\pp+\pp\cdot\frac{\rr}{r^3} \right) \\
        &= \ii\hbar\left( 2\frac{\rr}{r^3}\cdot\pp - \frac{\rr}{r^3}\cdot\pp +\pp\cdot\frac{\rr}{r^3} \right) \\
        &= \ii\hbar\left( 2\frac{\rr}{r^3}\cdot\pp + \left[\pp, \frac{\rr}{r^3}\right] \right) \\
        &= \ii\hbar\left( 2\frac{\rr}{r^3}\cdot\pp - \ii\hbar\del\cdot\frac{\rr}{r^3} \right).
    \end{aligned}
    \end{equation}
    \hfill$\square$

\end{tcolorbox}

\begin{tcolorbox}[breakable, title={\textbf{例题:径向动量算符}}]
    \it\small
    \fbox{ \textbf{题目} }
    求解径向动量算符
    \begin{equation}
        p_r \coloneq \pp \cdot \frac{\rr}{r}
    \end{equation}
    在坐标表象下的表达式。

    \fbox{ \textbf{解答} }
    这一算符涉及动量与坐标的乘积,有$p_r^†=(\rr/r)⋅\pp≠p_r$,因此不是Hermite算符。
    首先进行Hermitian化,定义Hermitian的径向动量算符:
    \begin{equation}
        \hat{p}_r = \frac12(p_r+p_r^†) = \frac12 \left( \frac{\rr}{r}\cdot\pp + \pp\cdot\frac{\rr}{r} \right).
    \end{equation}
    这还不够简化,可以通过构造对易子,进一步化简。
    \begin{equation}
    \begin{aligned}
        &\quad\ \frac12 \left( \frac{\rr}{r}\cdot\pp + \pp\cdot\frac{\rr}{r} \right) \\
        &= \frac12 \left( 2\frac{\rr}{r}\cdot\pp - \frac{\rr}{r}\cdot\pp + \pp\cdot\frac{\rr}{r} \right) \\
        &= \frac12 \left( 2\frac{\rr}{r}\cdot\pp + \left[\pp, \frac{\rr}{r}\right] \right) \\
        &= \frac{\rr}{r}\cdot\pp - \frac12\ii\hbar\left(\del\cdot\frac{\rr}{r}\right) \\
        &= \frac{\rr}{r}\cdot\pp - \ii\hbar\frac{1}{r}.
    \end{aligned}
    \end{equation}
    注意:这里的$\del$并非动量在坐标表象下的表示,而是对坐标算符的求导运算,以上推导与表象无关;
    $\del\cdot(\rr/r)$的计算步骤如下:
    \begin{equation}
    \begin{aligned}
        &\quad\ \del\cdot\frac{\rr}{r} \\
        &= \pd_x \frac{x}{\sqrt{x^2+y^2+z^2}} + \pd_y \frac{y}{\sqrt{x^2+y^2+z^2}} + \pd_z \frac{z}{\sqrt{x^2+y^2+z^2}} \\
        &= \left(\frac{1}{r} - \frac{x^2}{r^3}\right) + \left(\frac{1}{r} - \frac{y^2}{r^3}\right) + \left(\frac{1}{r} - \frac{z^2}{r^3}\right) \\
        &= \frac{2}{r}.
    \end{aligned}
    \end{equation}
    现在,我们正式将$\pp$换为坐标表象下的表示,得到:
    \begin{equation}
        \hat{p}^{(\rr)}_r = -\ii\hbar\pd_r - \ii\hbar\frac{1}{r} = -\ii\hbar\left(\frac{1}{r}+\pd_r\right).
    \end{equation}
    \hfill$\square$

\end{tcolorbox}
