
\section{全同粒子}
\label{sec:identical_particles}

全同粒子(identical particles)指不可区分的粒子,包含基本粒子(光子、电子、质子、中子)以及由基本粒子所组成的粒子,如原子核、原子、分子等。
本章我们将简单讨论全同粒子体系的量子态描述,以及全同粒子的交换对称性所带来的特殊情况。

% =================================================
\subsection{全同粒子量子态的描述}
\label{subsec:ip_state_desc}

我们考虑一由两个全同粒子组成的体系,在$q$表象下,体系的量子态$\ket{\Psi}$应同时由两个粒子各自的量子态$\ket{\psi}_1,\ket{\psi}_2$描述:
\begin{equation}
    \braket{\psi_{q_1},\psi_{q_2}}{\Psi} = \Psi_{q_1, q_2},
\end{equation}
这里的量子态$\ket{\psi_{q_1},\psi_{q_2}}=\ket{\psi_{q_1}}_1 \ket{\psi_{q_2}}_2$表示两个粒子中第一个处于$\ket{\psi_{q_1}}$,第二个处于$\ket{\psi_{q_2}}$的状态。
我们将看到,全同粒子体系的交换对称性将对体系的量子态提出一定要求,从而限制全同粒子体系实际波函数的形式。

% =================================================
\subsection{\texorpdfstring{交换对称性\quad Pauli不相容原理}{交换对称性  Pauli不相容原理}}
\label{subsec:ip_sym}

\begin{tcolorbox}
    全同粒子体系的一大特点是\emph{“不可分辨”}。
    这就是说,任何可观测量的测值概率分布在交换两个粒子时保持不变,称为\emph{“全同性原理”},是量子力学的基本公设之一。
\end{tcolorbox}

我们考虑置换算符$P_{ij}$,其作用在于交换粒子$i,j$的全部坐标:
\begin{equation}
    P_{ij}\Psi(\cdots,q_i,\cdots,q_j,\cdots) = \Psi(\cdots,q_j,\cdots,q_i,\cdots),
\end{equation}
交换前后的量子态的力学量测值分布不应有变化,因此$\ket{\Psi}$与$P_{ij}\ket{\Psi}$最多只应当相差常数$C$。
我们设$P_{ij}\ket{\Psi}=C\ket{\Psi}$,并对$\ket{\Psi}$作用两次置换算符:
\begin{equation}
    P_{ij}^2\ket{\Psi} = C^2 \ket{\Psi},
\end{equation}
作用两次后,波函数应当复原,因此$C^2=1$,于是我们知道置换算符的本征态有两个,分别满足
\begin{equation}
\begin{aligned}
    P_{ij}\ket{\Psi}_{\rm{S}} &=  \ket{\Psi}_{\rm{S}}, \\
    P_{ij}\ket{\Psi}_{\rm{A}} &= -\ket{\Psi}_{\rm{A}}, \\
\end{aligned}
\end{equation}
前者的下标为Symmetric,称为\emph{对称量子态};
后者的下标为Anti-symmetric,称为\emph{反对称量子态}。

实验表明,全同粒子体系的量子态交换对称性与粒子自旋有着确定的联系。
凡自旋为整数倍$\hbar$的粒子,量子态是交换对称的,如$\pi$介子($s=0$)、$\alpha$粒子($s=0$)、光子($s=\hbar$),称为\emph{Bose子(Bosen)};
凡自旋为半奇数倍$\hbar$的粒子,量子态是交换反对称的,如电子、质子和中子($s=\hbar/2$),称为\emph{Fermi子(Fermion)}。

满足交换对称/反对称原理的量子态如何构建?
我们考虑二粒子双本征态情形,设两个粒子能够分别处于单粒子态$\ket{\psi_{q_1}},\ket{\psi_{q_2}}$。

\begin{tcolorbox}
\textbf{Bose子}\quad
对于Bose子,量子态满足交换对称性,若$q_1\neq q_2$,我们可以将量子态写作归一化对称形式
\begin{equation}
    \ket{\psi_{q_1},\psi_{q_2}}_{\rm{S}}
    = \frac{1}{\sqrt{2}} (1+P_{12}) \left[\ket{\psi_{q_1}}_1 \ket{\psi_{q_2}}_2\right]
    = \frac{1}{\sqrt{2}} \left[\ket{\psi_{q_1}}_1 \ket{\psi_{q_2}}_2 + \ket{\psi_{q_2}}_1 \ket{\psi_{q_1}}_2\right],
\end{equation}
若$q_1=q_2=q$,则
\begin{equation}
    \ket{\psi_{q},\psi_{q}}_\rm{S} = \ket{\psi_q}_1 \ket{\psi_q}_2.
\end{equation}
\end{tcolorbox}

\begin{tcolorbox}
\textbf{Fermi子}\quad
对于Fermi子,量子态满足交换反对称性,若$q_1\neq q_2$,有
\begin{equation}
    \ket{\psi_{q_1},\psi_{q_2}}_{\rm{A}}
    = \frac{1}{\sqrt{2}} (1-P_{12}) \left[\ket{\psi_{q_1}}_1 \ket{\psi_{q_2}}_2\right]
    = \frac{1}{\sqrt{2}}
    \begin{vmatrix}
        \ket{\psi_{q_1}}_1 & \ket{\psi_{q_1}}_2 \\
        \ket{\psi_{q_2}}_1 & \ket{\psi_{q_2}}_2
    \end{vmatrix},
\end{equation}
这种形式称为\emph{Slater行列式};
若$q=q_1=q_2$,则
\begin{equation}
    \ket{\psi_q,\psi_q}_{\rm{A}} = 0.
\end{equation}
这说明\emph{两个全同Fermi子不能同时处于同一个单粒子态,称为Pauli不相容原理。}
\end{tcolorbox}


\begin{tcolorbox}[breakable, colframe=blue, colback=blue!10, title={\textbf{Bose, Fermi与经典体系粒子的组合数计算}}]

    一个常见的问题是计算Bose, Fermi与经典体系粒子的组合数:
    $n$\emph{个全同粒子处于}$m$\emph{个可能的单粒子态,对于Bose子, Fermi子与经典粒子,可能有几种组合方式?}

    对于经典粒子,粒子是完全可分辨的,因此粒子间互不相关,可能的组合数有
    \begin{equation}
        N_{\rm{classical}} = m^n.
    \end{equation}

    对于Bose子,粒子不可分辨,可以用隔板法求取可能的组合数:
    设置$m-1$个隔板,与$n$个粒子混合排列,这样$n$个粒子就被隔板分至$m$个态上,有$(n+m-1)!$种方式。
    隔板与粒子分别不可分辨,因此除以各自排列数$(m-1)!$与$n!$
    \begin{equation}
        N_{\rm{Bose}} = \frac{(n+m-1)!}{(m-1)! n!}.
    \end{equation}

    对于Fermi子,粒子不可分辨,且是互斥的,于是从$m\ (m>n)$个态中选取$n$个分别放入一个粒子即可,可能的组合数:
    \begin{equation}
        N_{\rm{Fermi}} = P_m^n = \frac{m!}{n!(m-n)!}.
    \end{equation}

\end{tcolorbox}
