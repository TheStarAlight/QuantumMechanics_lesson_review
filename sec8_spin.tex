\section{自旋}
\label{sec:spin}

% =================================================
\subsection{自旋-1/2体系的Pauli表象}
\label{subsec:spin_single_particle}

电子除了空间上的三个自由度,还有一个内禀(intrinsic)属性——\emph{自旋(spin)},用$\bm{s}$表示。
根据Stern-Gerlach实验,电子的自旋投影具有二值性,即只能取$s_z=+\hbar/2$或$s_z=-\hbar/2$。

自旋具有角动量的特征,因此我们假定其具备与角动量类似的量子性质,其对易关系满足:
\begin{equation}
    [s_i,s_j]=\ii\hbar\epsilon_{ijk}s_k.
\end{equation}
再引入无量纲自旋Pauli算符
\begin{equation}
    \bm{\sigma}=\bm{s}/\frac{\hbar}{2},
\end{equation}
对易关系成为:
\begin{equation}
    [\sigma_i,\sigma_j]=2\ii\epsilon_{ijk}\sigma_k.
\end{equation}
其还满足
\begin{equation}
    \sigma_i^2 = 1, \quad
    \sigma_i\sigma_j = \ii \epsilon_{ijk}\sigma_k \quad (i\ne j).
\end{equation}

Pauli表象选取$s_z=\pm\hbar/2$(即$\ket{\up},\ket{\down}$)的本征态作为描述系统状态的基矢。
在这一表象下,Pauli算符的各个分量为:
\begin{equation}
    \sigma_z = \begin{pmatrix}1& 0\\0&-1\end{pmatrix}, \quad
    \sigma_x = \begin{pmatrix}0& 1\\1& 0\end{pmatrix}, \quad
    \sigma_y = \begin{pmatrix}0&-\ii\\ \ii& 0\end{pmatrix}.
\end{equation}

\begin{tcolorbox}[breakable, title={\textbf{例题}}]
    \paragraph{题目} \textit{(2019年考研·六)}
    电子处于沿$z$轴方向,磁感应强度为$B_0 \cos\omega t$的时变磁场中。零时刻测得其自旋$x$分量为$+\hbar/2$。
    求$t$时刻的量子态。

    \paragraph{解答}
    我们在Pauli表象下研究这个问题。

    首先求出零时刻量子态在Pauli表象下的表示$\chi_0^{(s_z)}$,其为$\sigma_x$的本征值为$+1$的本征态。
    $\sigma_x$的矩阵形式为
    \begin{equation}
        \sigma_x^{(s_z)} = \begin{pmatrix} & 1 \\ 1 & \end{pmatrix},
    \end{equation}
    于是得到零时刻量子态
    \begin{equation}
        \chi_0^{(s_z)} = \frac{1}{\sqrt{2}} \begin{pmatrix} 1 \\ 1 \end{pmatrix}.
    \end{equation}

    接下来,我们写出Hamiltonian:
    \begin{equation}
        H = -\mu_{\rm{B}} B(t) \sigma_z = - \underbrace{\frac{e\hbar}{2m_e} B_0}_{E_0} \cos\omega t \sigma_z,
    \end{equation}
    其中$\mu_{\rm{B}}$是Bohr磁子。

    量子态的演化满足Schrödinger方程,我们设$\chi^{(s_z)}(t)=[\chi_a(t), \chi_b(t)]^T$,那么有
    \begin{equation}
        \ii\hbar\pd_t \begin{pmatrix}\chi_a \\ \chi_b \end{pmatrix}
        = H \begin{pmatrix} \chi_a \\ \chi_b \end{pmatrix}
        = \begin{pmatrix} -E_0 \cos\omega t \chi_a \\ E_0 \cos\omega t \chi_b \end{pmatrix},
    \end{equation}
    于是得到两个分离的常微分方程
    \begin{equation}
        \ii\hbar\pd_t \chi_a = -E_0 \cos\omega t \chi_a,\quad
        \ii\hbar\pd_t \chi_b =  E_0 \cos\omega t \chi_b,
    \end{equation}
    解之,并用初条件$\chi_a(0)=1/\sqrt2,\ \chi_b(0)=1/\sqrt2$定解,就得到$t$时刻的量子态
    \begin{equation}
        \chi_a(t) = \frac{1}{\sqrt{2}} \ee^{\ii\frac{E_0}{\hbar\omega}\sin\omega t},\quad
        \chi_b(t) = \frac{1}{\sqrt{2}} \ee^{-\ii\frac{E_0}{\hbar\omega}\sin\omega t}.
    \end{equation}
    \hfill$\square$
\end{tcolorbox}

在Hamiltonian不含时的情形中,若Hamiltonian能表为$\bm{\sigma}$的分量的线性组合(例如:$H=E_0\sigma_i$),可利用结论
\begin{equation}
    \label{eq:spin_exp_pauli}
    \ee^{\ii \bm{\alpha}\cdot\bm{\sigma}} = \cos\alpha + \ii \sin\alpha (\hat{\bm{\alpha}}\cdot\bm{\sigma}),
\end{equation}
而无需变换至$\sigma_i$表象。
该结论可利用$\ee^{\ii \bm{\alpha}\cdot\bm{\sigma}}$的Taylor展开式证明:
\begin{equation}
\begin{aligned}
    \ee^{\ii \bm{\alpha}\cdot\bm{\sigma}}
    &= \sum_{n=0}^\infty \frac{(\ii \bm{\alpha}\cdot\bm{\sigma})^n}{n!}\\
    &= \sum_{n=0}^\infty \frac{(-1)^n (\bm{\alpha}\cdot\bm{\sigma})^{2n}}{(2n)!} + \ii \sum_{n=0}^\infty \frac{(-1)^n (\bm{\alpha}\cdot\bm{\sigma})^{2n+1}}{(2n+1)!}\\
    &= \sum_{n=0}^\infty \frac{(-1)^n \alpha^{2n}}{(2n)!} + \ii (\hat{\alpha}\cdot\sigma) \sum_{n=0}^\infty \frac{(-1)^n \alpha^{2n+1}}{(2n+1)!}\\
    &= \cos\alpha + \ii \sin\alpha (\hat{\alpha}\cdot\bm{\sigma}).
\end{aligned}
\end{equation}

\begin{tcolorbox}[breakable, title={\textbf{例题:Larmor进动}}]
    \paragraph{题目} \textit{(2023年考研·五)}
    电子处于沿$x$轴方向,磁感应强度为$B_0$的静磁场中。零时刻测得其自旋$z$分量为$+\hbar/2$。
    求:\\
    (1) $t$时刻的量子态;\\
    (2) $t$时刻电子自旋各个分量的期望值。

    \paragraph{解答}
    Hamiltonian为
    \begin{equation}
        H = -\mu_{\rm{B}} B_0 \sigma_x = - \underbrace{\frac{e\hbar}{2m_e} B_0}_{E_0} \sigma_x,
    \end{equation}
    则$0\rightarrow t$的时间演化算符为
    \begin{equation}
        U(t,0) = \ee^{-\ii H t/\hbar} = \ee^{\ii E_0 \sigma_x t/\hbar},
    \end{equation}
    $t$时刻与零时刻的量子态通过时间演化算符相联系「式\eqref{eq:spin_exp_pauli}」:
    \begin{equation}
        \ket{\psi(t)} = U(t,0) \ket{\psi(0)} = \ee^{\ii E_0 \sigma_x t/\hbar} \ket{\psi_0}
        = \cos(E_0 t/\hbar)\ket{\psi_0} + \ii\sin(E_0 t/\hbar) \sigma_x \ket{\psi_0}.
    \end{equation}

    现在,我们在Pauli表象下写出$\ket{\psi(0)}$和$\ket{\psi(t)}$的具体表式。
    零时刻态为$s_z=\hbar/2$的本征态:
    \begin{equation}
        \psi^{(s_z)}(0) = \begin{pmatrix} 1 \\ 0 \end{pmatrix},
    \end{equation}
    代入上式,我们得到$t$时刻态的表示为
    \begin{equation}
        \psi^{(s_z)}(t) = \cos(E_0 t/\hbar)\begin{pmatrix} 1 \\ 0 \end{pmatrix} + \sin(E_0 t/\hbar) \underbrace{\begin{pmatrix} 0 & 1 \\ 1 & 0 \end{pmatrix}}_{\sigma_x} \begin{pmatrix} 1 \\ 0 \end{pmatrix}
        = \begin{pmatrix} \cos(E_0 t/\hbar) \\ \ii\sin(E_0 t/\hbar) \end{pmatrix}.
    \end{equation}

    自旋各分量的期望值为:
    \begin{equation}
    \begin{aligned}
        \expval{s_z}(t) &= \mel{\psi(t)}{s_z}{\psi(t)} \\
        &= \frac{\hbar}{2} \begin{pmatrix} \cos(E_0 t/\hbar) & -\ii\sin(E_0 t/\hbar) \end{pmatrix} \begin{pmatrix} 1 & \\ & -1 \end{pmatrix} \begin{pmatrix} \cos(E_0 t/\hbar) \\ \ii\sin(E_0 t/\hbar) \end{pmatrix} \\
        &= \frac{\hbar}{2} \left(\cos^2(E_0 t/\hbar) - \sin^2(E_0 t/\hbar)\right)
        = \frac{\hbar}{2} \cos(2E_0 t/\hbar); \\
        \expval{s_y}(t) &= \mel{\psi(t)}{s_y}{\psi(t)} = \frac{\hbar}{2} \sin(2E_0 t/\hbar)0; \\
        \expval{s_x}(t) &= \mel{\psi(t)}{s_x}{\psi(t)} = 0.
    \end{aligned}
    \end{equation}
    这里需要注意,写出左矢的行向量时,应对系数取复共轭。\\
    事实上,本题所述的就是Larmor进动现象——磁矩在外磁场中绕磁场方向进动。
    \hfill $\square$
\end{tcolorbox}

\begin{tcolorbox}[breakable, title={\textbf{趣题:量子Zeno效应}}]
    \paragraph{题目} \textit{(2025年考研·七)}
    电子处于沿$z$方向的磁场中,自旋Hamiltonian写作
    \begin{equation}
        H = \frac{e\hbar}{2m_e} B_0 σ_z = \hbar ω σ_z.
    \end{equation}
    考虑$[0,T]$时间段内重复沿着$x$轴方向测量自旋的取值,第一次在$t=0$时刻测量,之后每经过$Δt=T/N$时测量一次,在$t=T$时测量最后一次(第$N+1$次)。求:\\
    (1) 对于$N=1$,若首次测量时得到$σ_x=+1$的结果,在$t=T$时刻第二次测量中仍然得到$σ_x=+1$的结果的概率;\\
    (2) 若首次测量时得到$σ_x=+1$的结果,进行$N$次测量后,在最后一次测量中仍然得到$σ_x=+1$的结果的概率;\\
    (3) 对于测量次数$N→∞$,即相邻两次测量的时间间隔$Δt→0$(相当于达成了无间断的连续测量),题(1)的结论为何?反映了什么现象?
    \vspace{1em}

    \paragraph{解答}

    (1)
    首先,写出$t$时刻量子态的表达式,应用式\eqref{eq:spin_exp_pauli},有:
    \begin{equation}
        \ket{ψ(t)} = U(t,0)\ket{ψ(0)} = \ee^{-\ii ω σ_z t}\ket{ψ(0)} = \cos(ωt)\ket{ψ(0)} - \ii\sin(ωt)σ_z\ket{ψ(0)}.
    \end{equation}
    为方便起见,我们选取$σ_x$表象,以$σ_x=+1,-1$的自旋态为基底$↑,↓$。
    $σ_z$算符在该表象下是非对角的,需要计算其表示。
    一种做法是在熟悉的$σ_z$表象下求取矩阵元。
    我们在$σ_z$算符的左右侧各插入一组依$σ_x$本征态展开的完备性关系$\mathbb{1}=\ketbra{↑_x}+\ketbra{↓_x}$:
    \begin{equation}
        σ_z = \sum_{i=1,2} \sum_{j=1,2} \ket{σ_{xi}}\mel{σ_{xi}}{σ_z}{σ_{xj}}\bra{σ_{xj}},
    \end{equation}
    其中$\ket{σ_{x1}},\ket{σ_{x2}}$分别对应$\ket{↑},\ket{↓}$。
    $σ_z$在$σ_x$表象下的矩阵元就是
    \begin{equation}
        [σ_z^{(σ_x)}]_{ij} = \mel{σ_{xi}}{σ_z}{σ_{xj}},
    \end{equation}
    在$σ_z$表象下显式写出$\ket{↑_x},\ket{↓_x}$和$σ_z$即可求得各矩阵元,例如,对应$i=1,j=2$的矩阵元为
    \begin{equation}
        [σ_z^{(σ_x)}]_{12} = \mel{↑_x}{σ_z}{↓_x} =
        \begin{pmatrix}1/\sqrt{2} & 1/\sqrt{2}\end{pmatrix}
        \begin{pmatrix}1 \\ & -1\end{pmatrix}
        \begin{pmatrix}1/\sqrt{2} \\ -1/\sqrt{2}\end{pmatrix}
        =1.
    \end{equation}
    得到$σ_z$在$σ_x$表象下的表式
    \begin{equation}
        σ_z^{(σ_x)} = \begin{pmatrix}&1\\1\end{pmatrix}.
    \end{equation}
    代入$\ket{ψ(t)}$的表式以及$\ket{ψ(0)}=\ket{↑_x}$,即得
    \begin{equation}
        ψ^{(σ_x)}(t) = \left[\begin{pmatrix}\cos ωt\\&\cos ωt\end{pmatrix} + \begin{pmatrix}&-\ii\sin ωt\\-\ii\sin ωt&\end{pmatrix}\right]\begin{pmatrix}1 \\ 0\end{pmatrix} = \begin{pmatrix}\cos ωt \\ -\ii\sin ωt\end{pmatrix}.
    \end{equation}
    所以,在$t=T$时刻的第二次测量,得到$σ_x=+1$的概率为$\cos^2 ωT$,得到$σ_x=-1$的概率为$\sin^2 ωT$。

    题(2)还需要用到初态为$\ket{↓}$的情形。
    用同样的方法可得,初态为$\ket{↓}$时,$t=T$时刻的第二次测量,得到$σ_x=+1$的概率为$\sin^2 ωT$,得到$σ_x=-1$的概率为$\cos^2 ωT$。
    \vspace{1em}

    (2)
    本题涉及到了多次测量。
    测量作为一个与外部环境交互的过程,测量后系统的态已经不再是一个纯态,而是由不同的纯态按测量概率组成的``混合态''。
    对于本题所述的情况,可以如下分析:\\
    1. 系统的初态为$\ket{↑_x}$;\\
    2. 经过$Δt$时间演化,并测量$σ_x$后,系统坍缩至$P_{1↑}$概率为$\ket{↑_x}$、$P_{1↓}$概率为$\ket{↓_x}$的纯态组成的混合态,概率分别为
    \begin{equation}
        P_{1↑} = \cos^2(ωΔt) = c^2,\quad P_{1↓} = \sin^2(ωΔt) = s^2,
    \end{equation}
    这里我们使用了$c:=\cos(ωΔt), s:=\sin(ωΔt)$的简写。\\
    3. 再经过$Δt$时间演化后,这$P_{1↑}$概率的$\ket{↑_x}$与$P_{1↓}$概率的$\ket{↓_x}$分别独立演化至$\ket{↑_x}$与$\ket{↓_x}$的叠加态,在$t=2Δt$时的第二次测量后又按一定概率坍缩至两种态的混合态,概率为
    \begin{equation}
        P_{2↑} = P_{1↑} c^2 + P_{1↓} s^2,\quad P_{2↓} = P_{1↓} c^2 + P_{1↑} s^2.
    \end{equation}
    4. 据此我们可以得到第$n+1$次与第$n$次测量所得概率分布的关系:
    \begin{equation}
        P_{n+1,↑} = P_{n↑} c^2 + P_{n↓} s^2,\quad P_{n+1,↓} = P_{n↓} c^2 + P_{n↑} s^2.
    \end{equation}

    于是,问题变为求解这一关于$P_{n,↑},P_{n,↓}$的迭代方程。
    我们注意到,由于每一次测量的结果只能是$↑$或$↓$,因此必然有$P_{n↑}+P_{n↓}=1$。
    利用这一关系,可以在迭代方程中$P_{n+1,↑}$的表式中消去$P_{n↓}$,从而解耦迭代方程:
    \begin{equation}
        P_{n+1,↑} = P_{n↑}(c^2-s^2) + s^2.
    \end{equation}
    这一迭代方程可以通过构造等比数列的方式求解。
    具体来说,我们令
    \begin{equation}
        R_n := P_{n↑} + 𝒞,
    \end{equation}
    且$\{R_n\}$是公比为$(c^2-s^2)$的等比数列。
    利用
    \begin{equation}
        \frac{R_{n+1}}{R_n} = \frac{P_{n+1,↑}+𝒞}{P_{n↑}+𝒞} = c^2-s^2,
    \end{equation}
    解得$𝒞=-1/2$。
    于是,根据$P_{1↑}=1$,我们可以得到
    \begin{equation}
        P_{N+1,↑} = \frac12(c^2-s^2)^n + \frac12 = \frac12+\frac12\cos^n(2ωΔt).
    \end{equation}
    \vspace{1em}

    (3)
    $Δt→0$时,显然有$\cos^n(2ωΔt)→1$,于是
    \begin{equation}
        P_{N+1,↑} → 1.
    \end{equation}
    这表明,倘若以足够高的频率对一个量子系统进行观测,其状态将``冻结''于其最初的状态。
    这种现象称为\emph{量子Zeno效应}。

    事实上,对量子态进行观测的过程等价于将量子态与环境进行一次耦合,从而使其完全退相干的过程,因此,这一事实也可以理解为,假如一个量子系统与环境有很强的耦合作用,那么其将一直处于其初始的状态。

    \hfill $\square$
\end{tcolorbox}


% =================================================
\subsection{双自旋-1/2粒子}
\label{subsec:spin_double_particle}

描述双自旋-1/2粒子的量子态,有两种表象:
\begin{itemize}
    \item \emph{非耦合表象}:选取$s_{1z}, s_{2z}$为好量子数;
    \item \emph{耦合表象}:设$\bm{S}=\bm{s_1}+\bm{s_2}$为合角动量,选取$S^2, S_z$为好量子数。
\end{itemize}

非耦合表象选取的基底为
\begin{equation}
    \ket{\sigma_{1z}\sigma_{2z}} =\quad \ket{\up\up}, \quad \ket{\up\down}, \quad \ket{\down\up}, \quad \ket{\down\down}.
\end{equation}
直积态的矩阵表示可用Kroncker积计算,注意Kronecker积的运算规则:
\begin{equation}
    A \otimes B =
    \begin{pmatrix}
        A_{11}B & A_{12}B & \cdots \\
        A_{21}B & A_{22}B & \cdots \\
        \vdots  & \vdots  & \ddots \\
    \end{pmatrix}
\end{equation}

耦合表象选取的基底为
\begin{equation}
    \ket{\sigma \sigma_z} = \quad \underbrace{\ket{11}, \quad \ket{10}, \quad \ket{1-1}}_{\text{triplet}}, \quad \underbrace{\ket{00}}_{\text{singlet}}.
\end{equation}
由于
\begin{equation}
    \bm{S}^2 = \frac{\hbar^2}{2}(3+\bm{\sigma_1}\cdot\bm{\sigma_2}),
\end{equation}
可在非耦合表象下求取$\bm{\sigma_1}\cdot\bm{\sigma_2}$的本征值为$1,1,1,-3$,分别对应
\begin{equation}
    \ket{11}=\ket{\up\up},\quad \ket{10}=\frac{1}{\sqrt{2}}(\ket{\up\down}+\ket{\down\up}),\quad \ket{1-1}=\ket{\down\down}, \quad \ket{00}=\frac{1}{\sqrt{2}}(\ket{\up\down}-\ket{\down\up}).
\end{equation}

对于两个全同Bose和Fermi子,由于对称性限制,无法取以上所有的态!
全同Bose子仅能取三重态$\ket{11}, \ket{10}, \ket{1-1}$,而全同Fermi子只能取单重态$\ket{00}$.

如果一个双粒子体系组成的量子态能表为单粒子态的直积,则称为可分离态,否则,称为纠缠态。
Bell基是一组完备的双粒子自旋-1/2系统的基底,是四个最大纠缠态:
\begin{equation}
    \ket{\psi^{\pm}}=\frac{1}{\sqrt{2}}(\ket{\up\down}\pm\ket{\down\up}),\quad
    \ket{\phi^{\pm}}=\frac{1}{\sqrt{2}}(\ket{\up\up}\pm\ket{\down\down}).
\end{equation}
