
\section*{前言}
\markboth{前言}{前言}
\addcontentsline{toc}{section}{前言}

作为物理学中的重要分支,量子力学和相对论一起构筑了现代物理学的理论基础。
回顾近代物理学的发展历程,我们看到,推动这两门理论的建立的是一些反直觉的物理现象:光速不变、黑体辐射、分立的原子光谱、……
二十世纪飘浮于物理学大厦上的两朵乌云,最终在量子力学与相对论的建立下烟消云散。
一百多年来,我们见证了量子力学与相对论这两门理论在科学领域取得的重大成就,它们为人类社会的发展做出了不可磨灭的贡献。

量子力学和相对论是反直觉的。
我们第一次听说“量子”“相对论”“波粒二象性”这些概念,大概不是在我们的中学课堂里,而是在某一本科普杂志或者某个网站上的一条新闻。
我们觉得神奇,大抵是因为这是一种不可理解、不可想象的现象——在量子力学中,物质既是粒子又是波动,在确定中暗含着本源的不确定性;在相对论中,时间不再是均匀的,而与空间浑然一体。
这些反直觉的现象与概念让我们心中的量子力学与相对论笼罩了神秘的色彩。

当我们打开量子力学的课本,这种神秘的色彩却化身迷惘与不安。
期待揭开量子力学的面纱,却早早被数学语言绕得晕头转向,哪有心情再厘清背后的物理原理?
我们的课程着迷于量子力学的数学语言,却忘记了数学语言的背后需要物理图像的支持。
许多同学对量子力学的印象,大概只剩下所剩无几的无限深方势阱、对易子与氢原子能级了。
在复习的时候,也难免存在不少困惑了。

时值期末,为了解答各位同学在量子力学方面的困惑,我撰写了这份简单的量子力学复习讲义,力求用简明、清晰的文字呈现量子力学的理论框架。
希望它能帮助各位平安度过期末。
预祝期末顺利!

\hfill \textit{Mingyu Zhu}

\hfill \textit{2024年1月}

\pagebreak

\section*{修订记录}
\markboth{}{修订记录}
\addcontentsline{toc}{section}{修订记录}

此前的版本亮点主要在于基本原理的阐述,但仍然存在不少不足之处,缺失时间演化、中心力场、自旋、角动量等部分的内容。
今年一月,经过一轮考研复习,笔者对量子力学的理论框架有了更深入的理解,借此补齐了这些缺失的章节,并适当补充了相关例题。
此外,还对部分章节的结构进行了优化和调整。
希望这份复习资料能更好地帮助读者回顾、理解量子力学的核心概念,掌握解题思路。

如有疑问、勘误与建议,敬请联系作者:\href{mailto:myzhu@stu.ecnu.edu.cn}{myzhu@stu.ecnu.edu.cn},或在公众号下留言。

\noindent \textit{顺祝~~~~时祺}

\hfill \textit{Mingyu Zhu}

\hfill \textit{2025年1月}
