
\section{习题解答}
\label{sec:exercises}


\begin{tcolorbox}[breakable, title={\textbf{两个对易子的计算}}]
    \paragraph{题目}
    计算如下几个对易子:
    \begin{enumerate}
        \item
            \begin{equation}
                [\rr,H], [\pp,H]
            \end{equation}
            其中$H=\pp^2/2m + V(\rr)$;
        \item
        \begin{equation}
                [p^2,\frac{1}{r}].
            \end{equation}
    \end{enumerate}

    \paragraph{解答}

    \paragraph{(1)}
    利用结论
    \begin{equation}
        [\pp,F(\rr,\pp)] = -\ii\hbar\del_{\rr}F(\rr,\pp),\quad [\rr,F(\rr,\pp)] = \ii\hbar\del_{\pp}F(\rr,\pp),
    \end{equation}
    可得
    \begin{equation}
    \begin{aligned}\relax
        [\rr, H]
        &= \ii\hbar\del_{\pp}\left[\pp^2/2m + V(\rr)\right] \\
        &= \ii\hbar\del_{\pp}\left[\pp^2/2m\right] \\
        &= \ii\hbar\pp/m;
    \end{aligned}
    \end{equation}
    \begin{equation}
    \begin{aligned}\relax
        [\pp, H]
        &= -\ii\hbar\del_{\rr}\left[\pp^2/2m + V(\rr)\right] \\
        &= -\ii\hbar\del V(\rr).
    \end{aligned}
    \end{equation}
    \hfill$\square$

    \paragraph{(2)}
    考虑按算符乘积拆分成
    \begin{equation}
        [p^2,\frac{1}{r}] = \pp\cdot[\pp,\frac{1}{r}] + [\pp,\frac{1}{r}]\cdot\pp.
    \end{equation}
    接下来先计算$[\pp,\frac{1}{r}]$:
    \begin{equation}
        [\pp,\frac{1}{r}] = -\ii\hbar\del\frac{1}{r} = \ii\hbar\frac{\rr}{r^3}.
    \end{equation}
    代入即得
    \begin{equation}
        \pp\cdot[\pp,\frac{1}{r}] + [\pp,\frac{1}{r}]\cdot\pp = \ii\hbar\left( \frac{\rr}{r^3}\cdot\pp+\pp\cdot\frac{\rr}{r^3} \right),
    \end{equation}
    这一表达式可以继续化简,但\emph{不能直接代入动量算符在坐标表象中的表达式} $\hat{\pp}^{(\rr)}=-\ii\hbar\del$\footnote{对易子的计算无关表象,因此不能直接代入,而要使用之前的结论,构造对易子来求解。}:
    \begin{equation}
    \begin{aligned}
        &\quad\ \ii\hbar\left( \frac{\rr}{r^3}\cdot\pp+\pp\cdot\frac{\rr}{r^3} \right) \\
        &= \ii\hbar\left( 2\frac{\rr}{r^3}\cdot\pp - \frac{\rr}{r^3}\cdot\pp +\pp\cdot\frac{\rr}{r^3} \right) \\
        &= \ii\hbar\left( 2\frac{\rr}{r^3}\cdot\pp + \left[\pp, \frac{\rr}{r^3}\right] \right) \\
        &= \ii\hbar\left( 2\frac{\rr}{r^3}\cdot\pp - \ii\hbar\del\cdot\frac{\rr}{r^3} \right).
    \end{aligned}
    \end{equation}
    \hfill$\square$

\end{tcolorbox}


\begin{tcolorbox}[breakable, title={\textbf{径向动量算符}}]
    \paragraph{题目}
    求解径向动量算符
    \begin{equation}
        p_r \coloneq \pp \cdot \frac{\rr}{r}
    \end{equation}
    在坐标表象下的表达式。

    \paragraph{解答}
    这一算符涉及动量与坐标的乘积,是非Hermite的。
    首先进行Hermite化,定义Hermite的径向动量算符:
    \begin{equation}
        \hat{p}_r = \frac12 \left( \frac{\rr}{r}\cdot\pp + \pp\cdot\frac{\rr}{r} \right).
    \end{equation}
    这还不够简化,可以通过构造对易子,进一步化简。
    \begin{equation}
    \begin{aligned}
        &\quad\ \frac12 \left( \frac{\rr}{r}\cdot\pp + \pp\cdot\frac{\rr}{r} \right) \\
        &= \frac12 \left( 2\frac{\rr}{r}\cdot\pp - \frac{\rr}{r}\cdot\pp + \pp\cdot\frac{\rr}{r} \right) \\
        &= \frac12 \left( 2\frac{\rr}{r}\cdot\pp + \left[\pp, \frac{\rr}{r}\right] \right) \\
        &= \frac{\rr}{r}\cdot\pp - \frac12\ii\hbar\left(\del\cdot\frac{\rr}{r}\right) \\
        &= \frac{\rr}{r}\cdot\pp - \ii\hbar\frac{1}{r}.
    \end{aligned}
    \end{equation}
    注意:这里的$\del$并非动量在坐标表象下的表示,而是对坐标算符的求导运算,以上推导与表象无关;
    $\del\cdot(\rr/r)$的计算步骤如下:
    \begin{equation}
    \begin{aligned}
        &\quad\ \del\cdot\frac{\rr}{r} \\
        &= \pd_x \frac{x}{\sqrt{x^2+y^2+z^2}} + \pd_y \frac{y}{\sqrt{x^2+y^2+z^2}} + \pd_z \frac{z}{\sqrt{x^2+y^2+z^2}} \\
        &= \left(\frac{1}{r} - \frac{x^2}{r^3}\right) + \left(\frac{1}{r} - \frac{y^2}{r^3}\right) + \left(\frac{1}{r} - \frac{z^2}{r^3}\right) \\
        &= \frac{2}{r}.
    \end{aligned}
    \end{equation}
    现在,我们正式将$\pp$换为坐标表象下的表示,得到:
    \begin{equation}
        \hat{p}^{(\rr)}_r = -\ii\hbar\pd_r - \ii\hbar\frac{1}{r} = -\ii\hbar\left(\frac{1}{r}+\pd_r\right).
    \end{equation}
    \hfill$\square$

\end{tcolorbox}

\begin{tcolorbox}[breakable, title={\textbf{氢原子能级叠加态}}]
    \paragraph{题目}
    氢原子在$t=0$时刻处于量子态($\ket{nlm}$表示量子数$n,l,m$的本征态)
    \begin{equation}
        \ket{\psi(0)} = \frac12\ket{210} - \frac{1}{\sqrt{2}}\ket{310} - \frac{1}{\sqrt{2}}\ket{21-1},
    \end{equation}
    求$H, l^2, l_z$三个力学量:
    \begin{enumerate}
        \item{在$t=0$时刻的测值概率分布及平均值;}
        \item{任意$t$时刻的测值概率分布及平均值。}
    \end{enumerate}

    \paragraph{解答}

    注意:这个量子态\emph{尚未归一化},我们先进行归一化:
    \begin{equation}
        \ket{\psi(0)} = \frac{1}{\sqrt{5}}\ket{210} - \sqrt{\frac25}\ket{310} - \sqrt{\frac25}\ket{21-1}.
    \end{equation}

    首先明确各个力学量的量子数与本征值的关系:
    \begin{itemize}
        \item{能量:我们不妨设基态($n=1$)时的能量为$E_0=-13.6\ \rm{eV}$,则主量子数为$n$的量子态的本征能量为$E_0/n^2$;}
        \item{角动量平方:角量子数为$l$的本征态对应的角动量平方大小为$l(l+1)\hbar^2$;}
        \item{角动量$z$分量:磁量子数为$m$的本征态对应的角动量$z$分量为$m\hbar$。}
    \end{itemize}

    根据统计诠释,测得$n=2$即$E_n=E_0/4$的概率由$\ket{210}$和$\ket{21-1}$贡献:
    \begin{equation}
        P_{E_2} = \abs{1/\sqrt{5}}^2 + \abs{-\sqrt{2/5}}^2 = 3/5,
    \end{equation}
    测得$n=3$即$E_n=E_0/9$的概率由$\ket{310}$贡献:
    \begin{equation}
        P_{E_3} = \abs{-\sqrt{2/5}}^2 = 2/5.
    \end{equation}
    能量平均值为
    \begin{equation}
        \expval{H} = \frac35 E_2 + \frac25 E_3 = \frac{7}{36} E_0.
    \end{equation}

    依此可得
    \begin{equation}
    \begin{aligned}
        P_{l=1} &= 1, \\
        \expval{l^2} &= 2\hbar^2; \\
        P_{m=0} &= 3/5, \\
        P_{m=-1} &= 2/5, \\
        \expval{l_z} &= \frac35 \cdot 0\hbar + \frac25 \cdot -\hbar = -\frac25\hbar.
    \end{aligned}
    \end{equation}

    对于含时情况,我们写出对应的时间相位
    \begin{equation}
        \ket{\psi(0)} = \left(\frac{1}{\sqrt{5}}\ket{210} - \sqrt{\frac25}\ket{21-1}\right) \ee^{-\ii E_0 t/4\hbar} - \sqrt{\frac25}\ket{310} \ee^{-\ii E_0 t/9\hbar}.
    \end{equation}
    在按前一题的方法计算测值概率时,只要将对应的系数乘以相应的量子相位即可。
    事实上,我们要注意到,在氢原子体系中,$H, l^2, l_z$三个力学量均与Hamiltonian量对易,因此是守恒量,因此无论系统是否处于单个能量本征态,其力学量测值概率分布均不随时间而变。
    因此$t>0$时刻的测值分布与平均值与$t=0$时完全一样。
    \hfill $\square$

\end{tcolorbox}

\begin{tcolorbox}[breakable, title={\textbf{自旋}}]
    \paragraph{题目}
    电子在$t=0$时刻处于$s_z=\hbar/2$的本征态上。
    其受到沿$x$方向匀强磁场$B$的作用,其Hamiltonian为
    \begin{equation}
        H = \frac{eB}{mc} s_x.
    \end{equation}
    求:
    \begin{enumerate}
        \item $s_z$表象下自旋算符三分量$s_x,s_y,s_z$的矩阵形式,及其对应关系;
        \item $t=0$时刻,$s_x,s_y,s_z$各自的平均值;
        \item $t$时刻,$s_x,s_y,s_z$各自的平均值。
    \end{enumerate}

    \paragraph{解答}

    研究量子力学问题,明确自己处于什么表象十分重要,并注意,有一些概念是与表象无关的(例如,对易子\footnote{尽管如此,有时对易子也可以在具体的表象中求解。})。

    这一题中,我们选择了$s_z$表象。

    \paragraph{(1)}

    书本上定义了与自旋有关的Pauli算符,在$\sigma_z$表象下,其表达式是
    \begin{equation}
        \sigma_x =
        \begin{bmatrix}
            0 & 1 \\ 1 & 0\\
        \end{bmatrix},\quad
        \sigma_y =
        \begin{bmatrix}
            0 & -\ii \\ \ii & 0\\
        \end{bmatrix},\quad
        \sigma_z =
        \begin{bmatrix}
            1 & 0 \\ 0 & -1\\
        \end{bmatrix},
    \end{equation}
    其与自旋算符的关系是
    \begin{equation}
        \bm{s} = \frac{\hbar}{2} \bm{\sigma}.
    \end{equation}
    这说明$s_z$与$\sigma_z$表象是同一个表象,无须进行表象变换。
    因此我们可以直接写出自旋算符各分量的表达式
    \begin{equation}
        s_x = \frac{\hbar}{2}
        \begin{bmatrix}
            0 & 1 \\ 1 & 0\\
        \end{bmatrix},\quad
        s_y = \frac{\hbar}{2}
        \begin{bmatrix}
            0 & -\ii \\ \ii & 0\\
        \end{bmatrix},\quad
        s_z = \frac{\hbar}{2}
        \begin{bmatrix}
            1 & 0 \\ 0 & -1\\
        \end{bmatrix}.
    \end{equation}

    \paragraph{(2)}

    我们在$s_z$表象下求解$s_x,s_y,s_z$的平均值。
    初始时刻,量子态在$s_z$表象下的表示为
    \begin{equation}
        \ket{\psi(0)} =
        \begin{bmatrix}
            \ket{\up} & \ket{\down}
        \end{bmatrix}
        \begin{bmatrix}
            1 \\ 0
        \end{bmatrix}.
    \end{equation}
    于是我们可以据此求解自旋各分量的平均值:
    \begin{equation}
        \expval{s_x} = \mel{\psi}{s_x}{\psi} = \frac{\hbar}{2}
        \begin{bmatrix}
            1 & 0
        \end{bmatrix}
        \begin{bmatrix}
            0 & 1 \\ 1 & 0\\
        \end{bmatrix}
        \begin{bmatrix}
            1 \\ 0
        \end{bmatrix}
        = 0;
    \end{equation}
    \begin{equation}
        \expval{s_y} = \mel{\psi}{s_y}{\psi} = \frac{\hbar}{2}
        \begin{bmatrix}
            1 & 0
        \end{bmatrix}
        \begin{bmatrix}
            0 & -\ii \\ \ii & 0\\
        \end{bmatrix}
        \begin{bmatrix}
            1 \\ 0
        \end{bmatrix}
        = 0;
    \end{equation}
    \begin{equation}
        \expval{s_z} = \mel{\psi}{s_z}{\psi} = \frac{\hbar}{2}
        \begin{bmatrix}
            1 & 0
        \end{bmatrix}
        \begin{bmatrix}
            1 & 0 \\ 0 & -1\\
        \end{bmatrix}
        \begin{bmatrix}
            1 \\ 0
        \end{bmatrix}
        = \frac{\hbar}{2}.
    \end{equation}

    \paragraph{(3)}

    考虑含时问题时,我们需要将量子态分解为各个能量本征态,并分别赋以时间相位。
    这一步骤并非一定要在能量表象中进行,而可以在任意表象下进行。
    我们仍然选择$s_z$表象。

    首先,我们求取Hamiltonian算符的本征值和本征矢量在$s_z$表象下的表示,这一求解过程可以在任意表象下进行。
    Hamiltonian算符在$s_z$表象下的表示为
    \begin{equation}
        H^{(s_z)} = \frac{eB}{mc} s_x^{(s_z)} = \frac{eB}{mc}
        \begin{bmatrix}
            0 & 1 \\ 1 & 0\\
        \end{bmatrix}.
    \end{equation}
    $s_z$表象下的定态\schrodinger 方程为
    \begin{equation}
        \underbrace{\frac{eB}{mc}
        \begin{bmatrix}
            0 & 1 \\ 1 & 0\\
        \end{bmatrix}}_{H^{(s_z)}}
        \begin{bmatrix}
            \alpha \\ \beta
        \end{bmatrix}
        = E
        \begin{bmatrix}
            \alpha \\ \beta
        \end{bmatrix},
    \end{equation}
    解之,得两个归一化能量本征态矢及其本征能量
    \begin{equation}
        \ket{E_+}^{(s_z)} =
        \begin{bmatrix}
            1/\sqrt{2} \\ 1/\sqrt{2}
        \end{bmatrix},\quad
        E_+ = \frac{eB}{mc};
    \end{equation}
    \begin{equation}
        \ket{E_-}^{(s_z)} =
        \begin{bmatrix}
            1/\sqrt{2} \\ -1/\sqrt{2}
        \end{bmatrix},\quad
        E_- = -\frac{eB}{mc}.
    \end{equation}

    现在,我们将$t=0$时刻的量子态投影至这两个能量本征态上:
    \begin{equation}
    \begin{aligned}
        & \quad\ \ket{\psi(0)} \\
        &= \ket{E_+} \underbrace{{\braket{E_+}{\psi(0)}}}_{\rm{coefficient\ of\ }\ket{E_+}} + \ket{E_-} \underbrace{\braket{E_-}{\psi(0)}}_{\rm{coefficient\ of\ }\ket{E_-}} \\
        &= \frac{1}{\sqrt{2}} \ket{E_+} + \frac{1}{\sqrt{2}} \ket{E_-},
    \end{aligned}
    \end{equation}

    为方便计,我们置$E_0 = eB/mc$,$t$时刻的量子态是
    \begin{equation}
    \begin{aligned}
        \ket{\psi(t)}
        &= \frac{\ee^{-\ii E_0 t/\hbar}}{\sqrt{2}} \ket{E_+} + \frac{\ee^{\ii E_0 t/\hbar}}{\sqrt{2}} \ket{E_-} \\
        &= \frac12 \left(\ee^{-\ii E_0 t/\hbar} + \ee^{\ii E_0 t/\hbar}\right) \ket{\up_{s_z}} + \frac12 \left(\ee^{-\ii E_0 t/\hbar} - \ee^{\ii E_0 t/\hbar}\right) \ket{\down_{s_z}},
    \end{aligned}
    \end{equation}
    在$s_z$表象下的表示为
    \begin{equation}
        \ket{\psi(t)}^{(s_z)} =
        \begin{bmatrix}
            \cos(E_0 t/\hbar) \\ -\ii\sin(E_0 t/\hbar).
        \end{bmatrix}
    \end{equation}

    现在,我们可以求解各个力学量的平均值。
    因为$[s_x,H]=0$,所以$s_x$是守恒量,其测值平均值不随时间改变。
    $s_y$的平均值是
    \begin{equation}
    \begin{aligned}
        \expval{s_y}(t)
        &= \mel{\psi(t)}{s_y}{\psi(t)} \\
        &= \frac{\hbar}{2}
        \begin{bmatrix}
            \cos(E_0 t/\hbar) & \ii\sin(E_0 t/\hbar)
        \end{bmatrix}
        \begin{bmatrix}
            0 & -\ii \\ \ii & 0\\
        \end{bmatrix}
        \begin{bmatrix}
            \cos(E_0 t/\hbar) \\ -\ii\sin(E_0 t/\hbar)
        \end{bmatrix} \\
        &= -\frac{\hbar}{2} \sin(2E_0 t/\hbar) \\
        &= -\frac{\hbar}{2} \sin(2 eBt/m\hbar c);
    \end{aligned}
    \end{equation}
    $s_z$的平均值是
    \begin{equation}
    \begin{aligned}
        \expval{s_z}(t)
        &= \mel{\psi(t)}{s_z}{\psi(t)} \\
        &= \frac{\hbar}{2}
        \begin{bmatrix}
            \cos(E_0 t/\hbar) & \ii\sin(E_0 t/\hbar)
        \end{bmatrix}
        \begin{bmatrix}
            1 & 0 \\ 0 & -1\\
        \end{bmatrix}
        \begin{bmatrix}
            \cos(E_0 t/\hbar) \\ -\ii\sin(E_0 t/\hbar)
        \end{bmatrix} \\
        &= \frac{\hbar}{2} \cos(2E_0 t/\hbar) \\
        &= \frac{\hbar}{2} \cos(2 eBt/m\hbar c).
    \end{aligned}
    \end{equation}
    \hfill $\square$

\end{tcolorbox}
